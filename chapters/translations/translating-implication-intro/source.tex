\documentclass[../../../main.tex]{subfiles}

\begin{document}

%%%%%%%%%%%%%%%%%%%%%%%%%%%%%%%%%%%%%%%%%
%%%%%%%%%%%%%%%%%%%%%%%%%%%%%%%%%%%%%%%%%
%%%%%%%%%%%%%%%%%%%%%%%%%%%%%%%%%%%%%%%%%
\chapter{Translating implication introduction}

In the last chapter, we looked at minimal logic proofs that end with the \startrule/ rule, and we learned how to translate that last step into an equivalent step in linear logic. In this chapter, we will learn how to translate the last step if it uses the implication introduction rule.


%%%%%%%%%%%%%%%%%%%%%%%%%%%%%%%%%%%%%%%%%
%%%%%%%%%%%%%%%%%%%%%%%%%%%%%%%%%%%%%%%%%
\section{Implication introduction}

In a minimal logic proof tree, the last step might introduce an implication with the \implIntro/ rule:

\begin{prooftree*}
  \hypo{}
  \ellipsis{}{}
  \infer1[\implElim/$^{a}$]{\nd{A \impl/ B}}
\end{prooftree*}

\noindent
In order to do this, the proof tree above must assume ``$\nd{A}$'' and then from that derive a ``$\nd{B}$''. That is, the tree above would have to look like this (omitting the ``$:: true$'' annotations):

\begin{prooftree*}
  \hypo{}
  \infer1[\startrule/]{\nd{A}^{a}}
  \ellipsis{}{\nd{B}}
  \infer1[\implElim/$^{a}$]{\nd{A \impl/ B}}
\end{prooftree*}

\noindent
What hypothetical judgment does this prove? It doesn't prove a hypothetical judgment at all. It proves a categorical judgment. Since the branch from ``$\nd{A}$'' down to ``$\nd{B}$'' is discharged, there are no open assumptions left in this tree. Hence, the judgment this tree proves is a categorical one:

\begin{equation*}
  \ndturnstile/ \nd{A \impl/ B}
\end{equation*}

\noindent
We can translate this to an equivalent linear logic judgment:

\begin{equation*}
  \ndturnstile/ \nd{A \impl/ B}
  \hskip 1cm \rightsquigarrow \hskip 1cm 
  \ndturnstile/~\bang/\nd{A}^{*} \lolli/ \nd{B}^{*}
\end{equation*}

\noindent
To prove this, we first start with the assumption:

\begin{prooftree*}
  \hypo{}
  \infer1[\startrule/]{\bang/\nd{A}^{*}}
\end{prooftree*}

\noindent
Then, we derive ``$\nd{B}^{*}$'':

\begin{prooftree*}
  \hypo{}
  \infer1[\startrule/]{\bang/\nd{A}^{*}}
  \ellipsis{}{\nd{B}^{*}}
\end{prooftree*}

\noindent
Once have done that, we discharge the branch and introduce the lolli:

\begin{prooftree*}
  \hypo{}
  \infer1[\startrule/]{\bang/\nd{A}^{* a}}
  \ellipsis{}{\nd{B}^{*}}
  \infer1[\lolliIntro/$^{a}$]{\bang/\nd{A}^{*} \lolli/ \nd{B}^{*}}
\end{prooftree*}

\noindent
That is a proof of the categorical judgment we set out to prove:

\begin{equation*}
  \ndturnstile/~\bang/\nd{A}^{*} \lolli/ \nd{B}^{*}
\end{equation*}

\noindent
How does this derivation of ``$\nd{B}^{*}$'' from ``$\bang/\nd{A}^{*}$'' actually play out? We need to use the \bangDer/ rule or the \bangCopy/ rule to actually use the banged assumption. If we use the \bangCopy/ rule, the proof tree will look like this:

\begin{prooftree*}
  \hypo{}
  \infer1[\startrule/]{\bang/\nd{A}^{* a}}
  \infer1[\bangCopy/]{\nd{A}^{*}}
  \ellipsis{}{\nd{B}^{*}}
  \infer1[\lolliIntro/$^{a}$]{\bang/\nd{A}^{*} \lolli/ \nd{B}^{*}}
\end{prooftree*}

\noindent
If you prefer the \bangDer/ rule, it will look like this:

\begin{prooftree*}
  \hypo{}
  \infer1[\startrule/]{\bang/\nd{A}^{* b}}
  \hypo{}
  \infer1[\startrule/]{\nd{A}^{* a}}
  \ellipsis{}{\nd{B}^{*}}
  \infer2[\bangDer/$^{a}$]{\nd{B}^{*}}
  \infer1[\lolliIntro/$^{b}$]{\bang/\nd{A}^{*} \lolli/ \nd{B}^{*}}
\end{prooftree*}

\noindent
We can push the extraction of ``$\nd{A}^{*}$'' up higher though, like this:

\begin{prooftree*}
  \hypo{}
  \infer1[\startrule/]{\bang/\nd{A}^{* b}}
  \hypo{}
  \infer1[\startrule/]{\nd{A}^{* a}}
  \infer2[\bangDer/$^{a}$]{\nd{A}^{*}}
  \ellipsis{}{\nd{B}^{*}}
  \infer1[\lolliIntro/$^{b}$]{\bang/\nd{A}^{*} \lolli/ \nd{B}^{*}}
\end{prooftree*}

\noindent
In all of these variations, we prove the same judgment:

\begin{equation*}
  \ndturnstile/~\bang/\nd{A}^{*} \lolli/ \nd{B}^{*}
\end{equation*}

\noindent
The conclusion of the tree goes on the right side of the turnstile, and there are no open assumptions in the tree, so nothing goes on the left side of the turnstile.


%%%%%%%%%%%%%%%%%%%%%%%%%%%%%%%%%%%%%%%%%
%%%%%%%%%%%%%%%%%%%%%%%%%%%%%%%%%%%%%%%%%
\section{Implication introduction with context}

\noindent
In the above, we translated a proof that ended with \implIntro/, and there was no further context. To be more general, we need to do the same thing, but with more context above that last implication introduction step. So the judgment in minimal logic that we want to prove is going to look like this:

\begin{equation*}
  \nd{\context/} \ndturnstile/ \nd{A} \impl/ \nd{B}
\end{equation*}

\noindent
To prove this in minimal logic, we start with whatever the assumptions ``$\nd{\context/}$'' prove:

\begin{prooftree*}
  \hypo{}
  \infer[no rule]1{\nd{\context/}}
  \ellipsis{}{\Proof/}
  \ellipsis{}{}
\end{prooftree*}

\noindent
Then we introduce ``$\nd{A}$'':

\begin{prooftree*}
  \hypo{}
  \infer[no rule]1{\nd{\context/}}
  \ellipsis{}{\Proof/}
  \ellipsis{}{}
  \hypo{}
  \infer1[\startrule/]{\nd{A}}
  \infer[no rule]2{}
\end{prooftree*}

\noindent
Then we use those assumptions to derive ``$\nd{B}$'':

\begin{prooftree*}
  \hypo{}
  \infer[no rule]1{\nd{\context/}}
  \ellipsis{}{\Proof/}
  \ellipsis{}{}
  \hypo{}
  \infer1[\startrule/]{\nd{A}}
  \ellipsis{}{}
  \infer[no rule]2{}
  \ellipsis{}{\nd{B}}
\end{prooftree*}

\noindent
Finally, we discharge the assumption ``$\nd{A}$,'' and we introduce the implication ``$\nd{A} \impl/ \nd{B}$'':

\begin{prooftree*}
  \hypo{}
  \infer[no rule]1{\nd{\context/}}
  \ellipsis{}{\Proof/}
  \ellipsis{}{}
  \hypo{}
  \infer1[\startrule/]{\nd{A}^{a}}
  \ellipsis{}{}
  \infer[no rule]2{}
  \ellipsis{}{\nd{B}}
  \infer1[\implIntro/$^{a}$]{\nd{A} \impl/ \nd{B}}
\end{prooftree*}


%%%%%%%%%%%%%%%%%%%%%%%%%%%%%%%%%%%%%%%%%
%%%%%%%%%%%%%%%%%%%%%%%%%%%%%%%%%%%%%%%%%
\section{The translation}

To translate this proof into an equivalent linear logic proof, we first translate the hypothetical judgment:

\begin{equation*}
  \nd{\context/} \ndturnstile/ \nd{A} \impl/ \nd{B}
  \hskip 1cm \rightsquigarrow \hskip 1cm
  \bang/\nd{\context/}^{*} \ndturnstile/~\bang/\nd{A}^{*} \lolli/ \nd{B}^{*}
\end{equation*}

\noindent
So the hypothetical judgment we want to prove in linear logic is this:

\begin{equation*}
  \bang/\nd{\context/}^{*} \ndturnstile/~\bang/\nd{A}^{*} \lolli/ \nd{B}^{*}
\end{equation*}

\noindent
To prove this, we begin with whatever ``$\bang/\nd{\Gamma}^{*}$'' proves:

\begin{prooftree*}
  \hypo{}
  \infer[no rule]1{\bang/\nd{\context/}^{*}}
  \ellipsis{}{\Proof/}
  \ellipsis{}{}
\end{prooftree*}

\noindent
Then we introduce the assumption ``$\bang/\nd{A}^{*}$'':

\begin{prooftree*}
  \hypo{}
  \infer[no rule]1{\bang/\nd{\context/}^{*}}
  \ellipsis{}{\Proof/}
  \ellipsis{}{}
  \hypo{}
  \infer1[\startrule/]{\bang/\nd{A}^{*}}
  \infer[no rule]2{}
\end{prooftree*}

\noindent
Then we extract a copy of ``$\nd{A}^{*}$.'' For example, we can use the \bangCopy/ rule:

\begin{prooftree*}
  \hypo{}
  \infer[no rule]1{\bang/\nd{\context/}^{*}}
  \ellipsis{}{\Proof/}
  \ellipsis{}{}
  \hypo{}
  \infer1[\startrule/]{\bang/\nd{A}^{*}}
  \infer1[\bangCopy/]{\nd{A}^{*}}
  \infer[no rule]2{}
\end{prooftree*}

\noindent
Then we derive ``$\nd{B}^{*}$'':

\begin{prooftree*}
  \hypo{}
  \infer[no rule]1{\bang/\nd{\context/}^{*}}
  \ellipsis{}{\Proof/}
  \ellipsis{}{}
  \hypo{}
  \infer1[\startrule/]{\bang/\nd{A}^{*}}
  \infer1[\bangCopy/]{\nd{A}^{*}}
  \ellipsis{}{}
  \infer[no rule]2{}
  \ellipsis{}{\nd{B}^{*}}
\end{prooftree*}

\noindent
Finally, we discharge ``$\bang/\nd{A}^{*}$'' and introduce the lolli:

\begin{prooftree*}
  \hypo{}
  \infer[no rule]1{\bang/\nd{\context/}^{*}}
  \ellipsis{}{\Proof/}
  \ellipsis{}{}
  \hypo{}
  \infer1[\startrule/]{\bang/\nd{A}^{* a}}
  \infer1[\bangCopy/]{\nd{A}^{*}}
  \ellipsis{}{}
  \infer[no rule]2{}
  \ellipsis{}{\nd{B}^{*}}
  \infer1[\lolliIntro/$^{a}$]{\bang/\nd{A}^{*} \lolli/ \nd{B}^{*}}
\end{prooftree*}

\noindent
Alternatively, we can use the \bangDer/ rule:

\begin{prooftree*}
  \hypo{}
  \infer[no rule]1{\bang/\nd{\context/}^{*}}
  \ellipsis{}{\Proof/}
  \ellipsis{}{}
  \hypo{}
  \infer1[\startrule/]{\bang/\nd{A}^{* b}}
  \hypo{}
  \infer1[\startrule/]{\nd{A}^{* a}}
  \infer2[\bangDer/$^{a}$]{\nd{A}^{*}}
  \ellipsis{}{}
  \infer[no rule]2{}
  \ellipsis{}{\nd{B}^{*}}
  \infer1[\lolliIntro/$^{b}$]{\bang/\nd{A}^{*} \lolli/ \nd{B}^{*}}
\end{prooftree*}

\noindent
Whichever of these variants we choose, we end up proving the same hypothetical assumption:

\begin{equation*}
  \bang/\nd{\context/}^{*} \ndturnstile/~\bang/\nd{A}^{*} \lolli/ \nd{B}^{*}
\end{equation*}

\noindent
The only open assumptions are ``$\bang/\nd{\context/}^{*}$,'' since ``$\bang/\nd{A}$'' gets discharged, and the conclusion is the lolli ``$\bang/\nd{A}^{*} \lolli/ \nd{B}^{*}$,'' which is exactly what we set out to prove.


%%%%%%%%%%%%%%%%%%%%%%%%%%%%%%%%%%%%%%%%%
%%%%%%%%%%%%%%%%%%%%%%%%%%%%%%%%%%%%%%%%%
\section{The general translation}

The general last-step translation is the one that includes the context. The version above with no context is a special case, where the context is empty. That is, if we set ``$\nd{\context/}$'' to ``$\emptyassumptions/$,'' then this

\begin{equation*}
  \nd{\context/} \ndturnstile/ \nd{A} \impl/ \nd{B}
  \hskip 1cm \rightsquigarrow \hskip 1cm
  \bang/\nd{\context/}^{*} \ndturnstile/~\bang/\nd{A}^{*} \lolli/ \nd{B}^{*}
\end{equation*}

\noindent
becomes this:

\begin{equation*}
  \emptyassumptions/ \ndturnstile/ \nd{A} \impl/ \nd{B}
  \hskip 1cm \rightsquigarrow \hskip 1cm
  \emptyassumptions/ \ndturnstile/~\bang/\nd{A}^{*} \lolli/ \nd{B}^{*}
\end{equation*}

\noindent
Which is the same as this:

\begin{equation*}
  \ndturnstile/ \nd{A} \impl/ \nd{B}
  \hskip 1cm \rightsquigarrow \hskip 1cm
  \ndturnstile/~\bang/\nd{A}^{*} \lolli/ \nd{B}^{*}
\end{equation*}

\noindent
And that is the special case we covered first, when there is no context.

Here is the full translation for the last step (the implication introduction):

$$
\begin{prooftree}
  \hypo{}
  \infer[no rule]1{\nd{\context/}}
  \ellipsis{}{\Proof/}
  \ellipsis{}{}
  \hypo{}
  \infer1[\startrule/]{\nd{A}^{a}}
  \ellipsis{}{}
  \infer[no rule]2{}
  \ellipsis{}{\nd{B}}
  \infer1[\implIntro/$^{a}$]{\nd{A} \impl/ \nd{B}}
\end{prooftree}
\hskip 1cm \rightsquigarrow \hskip 1cm
\begin{prooftree}
  \hypo{}
  \infer[no rule]1{\bang/\nd{\context/}^{*}}
  \ellipsis{}{\Proof/}
  \ellipsis{}{}
  \hypo{}
  \infer1[\startrule/]{\bang/\nd{A}^{* a}}
  \infer1[\bangCopy/]{\nd{A}^{*}}
  \ellipsis{}{}
  \infer[no rule]2{}
  \ellipsis{}{\nd{B}^{*}}
  \infer1[\lolliIntro/$^{a}$]{\bang/\nd{A}^{*} \lolli/ \nd{B}^{*}}
\end{prooftree}
$$


%%%%%%%%%%%%%%%%%%%%%%%%%%%%%%%%%%%%%%%%%
%%%%%%%%%%%%%%%%%%%%%%%%%%%%%%%%%%%%%%%%%
\section{Summary}

In this chapter, we looked at minimal logic proofs that end with the \implIntro/ rule. We established a way to translate that last step from minimal logic into an equivalent form in linear logic. in minimal logic, if you introduce an implication of the form ``$\nd{A} \impl/ \nd{B}$,'' you derive ``$\nd{B}$'' from an assumed ``$\nd{A}$'' (along with any other context ``$\nd{\context/}$''). In an equivalent proof in linear logic, you derive ``$\nd{B}$'' from an assumed ``$\bang/\nd{A}^{*}$'' (along with any other context ``$\bang/\nd{\context/}^{*}$'').


\end{document}
