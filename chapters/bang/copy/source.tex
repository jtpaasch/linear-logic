\documentclass[../../../main.tex]{subfiles}
\begin{document}

%%%%%%%%%%%%%%%%%%%%%%%%%%%%%%%%%%%%%%%%%
%%%%%%%%%%%%%%%%%%%%%%%%%%%%%%%%%%%%%%%%%
%%%%%%%%%%%%%%%%%%%%%%%%%%%%%%%%%%%%%%%%%
\chapter{The copy rule}
 
 
%%%%%%%%%%%%%%%%%%%%%%%%%%%%%%%%%%%%%%%%%
%%%%%%%%%%%%%%%%%%%%%%%%%%%%%%%%%%%%%%%%%
\section{A simpler dereliction rule}

The dereliction rule leads to proof trees that are large and complex. There is a simpler version of a dereliction rule, which can be found in G. M. Bierman's doctoral thesis. We will call it the \vocab{bang copy} rule. The template looks like this:

\begin{prooftree*}
  \hypo{}
  \infer1[\startrule/]{!A :: true}  
  \infer1[\bangCopy/]{A :: true}
\end{prooftree*}
 
\noindent
This rule simply says that if you start with a bang, you can get a copy of it.

The usage instructions are these: each time you need a copy of a bang, first introduce the bang with the \startrule/ rule, then use the \bangCopy/ rule to extract a copy.


%%%%%%%%%%%%%%%%%%%%%%%%%%%%%%%%%%%%%%%%%
%%%%%%%%%%%%%%%%%%%%%%%%%%%%%%%%%%%%%%%%%
\section{Example}

Look at the proof tree we built in the last chapter: 


\begin{prooftree*}
  \hypo{}
  \infer1[\startrule/]{!t(p, s)}

  \hypo{}
  \infer1[\startrule/]{!t(p, s)}

  \hypo{}
  \infer1[\startrule/]{b(p)}

  \hypo{}
  \infer1[\startrule/]{t(p, s)^{a}}
  
  \infer2[\traderule/]{b(s)}
  \infer2[\bangDer/$^{a}$]{b(s)}

  \hypo{}
  \infer1[\startrule/]{t(s, p)}  
  
  \infer2[\traderule/]{b(p)}

  \hypo{}
  \infer1[\startrule/]{t(p, s)^{b}}
  
  \infer2[\traderule/]{b(s)}

  \infer2[\bangDer/$^{b}$]{b(s)}
\end{prooftree*}

\noindent
If we use the simpler \bangCopy/ rule, we can re-write the tree like this:

\begin{prooftree*}
  \hypo{}
  \infer1[\startrule/]{b(p)}
  
  \hypo{}
  \infer1[\startrule/]{!t(p, s)}
  \infer1[\bangCopy/]{t(p, s)}
  
  \infer2[\traderule/]{b(s)}
  
  \hypo{}
  \infer1[\startrule/]{t(s, p)}
  
  \infer2[\traderule/]{b(p)}

  \hypo{}
  \infer1[\startrule/]{!t(p, s)}
  \infer1[\bangCopy/]{t(p, s)}

  \infer2[\traderule/]{b(s)}

\end{prooftree*}

\noindent
This is obviously more compact, and much simpler. To understand how it works, let's rebuild this tree, from the start. First, let's start with ``$b(p)$'' and ``$!t(p, s)$'':

\begin{prooftree*}
  \hypo{}
  \infer1[\startrule/]{b(p)}
  
  \hypo{}
  \infer1[\startrule/]{!t(p, s)}
  
  \infer[no rule]2{}

\end{prooftree*}

\noindent
Now we want to use the \traderule/ with ``$b(p)$'' and ``$t(p, s)$.'' To do that, we first need to get a copy of ``$t(p, s)$,'' which we can do with the simplified \bangCopy/ rule:

\begin{prooftree*}
  \hypo{}
  \infer1[\startrule/]{b(p)}
  
  \hypo{}
  \infer1[\startrule/]{!t(p, s)}
  \infer1[\bangCopy/]{t(p, s)}
  
  \infer[no rule]2{}

\end{prooftree*}

\noindent
Now we use the \traderule/:

\begin{prooftree*}
  \hypo{}
  \infer1[\startrule/]{b(p)}
  
  \hypo{}
  \infer1[\startrule/]{!t(p, s)}
  \infer1[\bangCopy/]{t(p, s)}
  
  \infer2[\traderule/]{b(s)}

\end{prooftree*}
 
 \noindent
 Next, add the assumption ``$t(s, p)$'':
 
\begin{prooftree*}
  \hypo{}
  \infer1[\startrule/]{b(p)}

  \hypo{}
  \infer1[\startrule/]{!t(p, s)}
  \infer1[\bangCopy/]{t(p, s)}

  \infer2[\traderule/]{b(s)}

  \hypo{}
  \infer1[\startrule/]{t(s, p)}

  \infer[no rule]2{}

\end{prooftree*}
 
\noindent
Use the \traderule/ again:

\begin{prooftree*}
  \hypo{}
  \infer1[\startrule/]{b(p)}
  
  \hypo{}
  \infer1[\startrule/]{!t(p, s)}
  \infer1[\bangCopy/]{t(p, s)}
  
  \infer2[\traderule/]{b(s)}
  
  \hypo{}
  \infer1[\startrule/]{t(s, p)}
  
  \infer2[\traderule/]{b(p)}

\end{prooftree*} 

\noindent
Now we want to use a copy of the bang ``$!t(p, s)$'' again, so we simply add that as a new assumption:


\begin{prooftree*}
  \hypo{}
  \infer1[\startrule/]{b(p)}
  
  \hypo{}
  \infer1[\startrule/]{!t(p, s)}
  \infer1[\bangCopy/]{t(p, s)}
  
  \infer2[\traderule/]{b(s)}
  
  \hypo{}
  \infer1[\startrule/]{t(s, p)}
  
  \infer2[\traderule/]{b(p)}

  \hypo{}
  \infer1[\startrule/]{!t(p, s)}

  \infer[no rule]2{}

\end{prooftree*}

\noindent
Then use the \bangCopy/ rule to get a copy of it:

\begin{prooftree*}
  \hypo{}
  \infer1[\startrule/]{b(p)}
  
  \hypo{}
  \infer1[\startrule/]{!t(p, s)}
  \infer1[\bangCopy/]{t(p, s)}
  
  \infer2[\traderule/]{b(s)}
  
  \hypo{}
  \infer1[\startrule/]{t(s, p)}
  
  \infer2[\traderule/]{b(p)}

  \hypo{}
  \infer1[\startrule/]{!t(p, s)}
  \infer1[\bangCopy/]{t(p, s)}

  \infer[no rule]2{}

\end{prooftree*}

\noindent
And finally, use the \traderule/ again to derive the desired conclusion:

\begin{prooftree*}
  \hypo{}
  \infer1[\startrule/]{b(p)}
  
  \hypo{}
  \infer1[\startrule/]{!t(p, s)}
  \infer1[\bangCopy/]{t(p, s)}
  
  \infer2[\traderule/]{b(s)}
  
  \hypo{}
  \infer1[\startrule/]{t(s, p)}
  
  \infer2[\traderule/]{b(p)}

  \hypo{}
  \infer1[\startrule/]{!t(p, s)}
  \infer1[\bangCopy/]{t(p, s)}

  \infer2[\traderule/]{b(s)}

\end{prooftree*}


%%%%%%%%%%%%%%%%%%%%%%%%%%%%%%%%%%%%%%%%%
%%%%%%%%%%%%%%%%%%%%%%%%%%%%%%%%%%%%%%%%%
\section{Dereliction and copy}
 
The copy rule can actually be derived from the dereliction rule. Here is the dereliction rule:

\begin{prooftree*}
  \hypo{}
  \ellipsis{}{!A :: true}
  
  \hypo{}
  \infer1[\startrule/]{A :: true^{x}}
  \ellipsis{}{$B :: true$}
  
  \infer2[\bangDer/$^{x}$]{B :: true}
\end{prooftree*}

\noindent
This says that if we can show how to derive a conclusion ``$B :: true$'' from an un-banged copy of ``$!A :: true$,'' then we are licensed to infer that conclusion from the bang.

Well, we can use that to simply infer an un-banged copy of ``$A :: true$.'' Suppose we have derived a bang of the form ``$!A :: true$'':

\begin{prooftree*}
  \hypo{}
  \ellipsis{}{!A :: true}
\end{prooftree*}

\noindent
Now let's assume an un-banged copy of the bang:

\begin{prooftree*}
  \hypo{}
  \ellipsis{}{!A :: true}
  
  \hypo{}
  \infer1[\startrule/]{\fbox{$A :: true$}}
  
  \infer[no rule]2{}
\end{prooftree*}

\noindent
We'll discharge this in a moment, so let's mark it with a label, say ``$a$'':

\begin{prooftree*}
  \hypo{}
  \ellipsis{}{!A :: true}
  
  \hypo{}
  \infer1[\startrule/]{\fbox{$A :: true^{a}$}}
  
  \infer[no rule]2{}
\end{prooftree*}

\noindent
Now, what can we derive from this? Well, if we've assumed ``$A :: true$,'' then the one thing we can infer for sure is that ``$A :: true$'' follows! After all, ``$A :: true$'' holds whenever you've assumed ``$A :: true$.'' So we can use the dereliction rule to directly conclude ``$A :: true$'':

\begin{prooftree*}
  \hypo{}
  \ellipsis{}{!A :: true}
  
  \hypo{}
  \infer1[\startrule/]{A :: true^{a}}
  
  \infer2[\bangDer/$^{a}$]{\fbox{$A :: true$}}
\end{prooftree*}

\noindent
This shows that we can always derive an un-banged copy from a banged judgment, using the dereliction rule. So we can see how this reduces to the simpler dereliction rule:

$$
\begin{prooftree}
  \hypo{}
  \ellipsis{}{!A :: true}
  \hypo{}
  \infer1[\startrule/]{A :: true^{a}}
  \infer2[\bangDer/$^{a}$]{A :: true}
\end{prooftree}
\quad\rightsquigarrow\quad
\begin{prooftree}
  \hypo{}
  \infer1[\startrule/]{!A :: true}  
  \infer1[\bangCopy/]{A :: true}
\end{prooftree}
$$

\noindent
The \bangCopy/ rule is thus a \vocab{derived rule}.


%%%%%%%%%%%%%%%%%%%%%%%%%%%%%%%%%%%%%%%%%
%%%%%%%%%%%%%%%%%%%%%%%%%%%%%%%%%%%%%%%%%
\section{Summary}

The \bangCopy/ rule can be derived from the \bangDer/ rule, and it is much simpler. It says: if you need to use a bang, introduce it into the proof tree with the \startrule/, and then use the \bangCopy/ rule to extract a linear copy that you can use.


\end{document}
