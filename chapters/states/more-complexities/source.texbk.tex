\documentclass[../../../main.tex]{subfiles}
\begin{document}

%%%%%%%%%%%%%%%%%%%%%%%%%%%%%%%%%%%%%%%%%
%%%%%%%%%%%%%%%%%%%%%%%%%%%%%%%%%%%%%%%%%
%%%%%%%%%%%%%%%%%%%%%%%%%%%%%%%%%%%%%%%%%
\chapter{More complexities}

A system of states can have any number of loops and branches. And this means that transition systems can get fairly complex. Here are some examples of some other complexities.


%%%%%%%%%%%%%%%%%%%%%%%%%%%%%%%%%%%%%%%%%
%%%%%%%%%%%%%%%%%%%%%%%%%%%%%%%%%%%%%%%%%
\section{Two paths to the same endpoint}

Suppose that if Jo has nothing in her hand, she can pick up either the first or third nickel. Like this:

\begin{diagram}

  % State 0
  \draw (-2, -1) -- (2, -1) -- (2, 2) -- (-2, 2) -- (-2, -1);
  \coordinate[label=below:{\textbf{S}$_{0}$}] (s_0) at (0, -1);
  
    \node[o-point] (jo) [label=below:{Jo}] at (0, -0.25) {};
    \node[o-point] (n_1) [label=above:{$n_{1}$}] at (-1.25, 1) {};
    \node[o-point] (n_2) [label=above:{$n_{2}$}] at (0, 1.25) {};
    \node[o-point] (n_3) [label=above:{$n_{3}$}] at (1.25, 1) {};
  
  % State 1
  \draw[spaced-arrows,->] (2, 0.5) -- (4, 0.5);
  \draw (4, -1) -- (8, -1) -- (8, 2) -- (4, 2) -- (4, -1);
  \coordinate[label=below:{\textbf{S}$_{1}$}] (s_1) at (6, -1);

    \node[o-point] (jo_1) [label=below:{Jo}] at (6, -0.25) {};
    \node[o-point] (n_1_1) [label=above:{$n_{1}$}] at (4.75, 1) {};
    \node[o-point] (n_2_1) [label=above:{$n_{2}$}] at (6, 1.25) {};
    \node[o-point] (n_3_1) [label=above:{$n_{3}$}] at (7.25, 1) {};
  
    \draw[spaced-arrows,->] (jo_1) to node [fill=white] {holds} (n_1_1);

  % State 2
  \draw[spaced-arrows,->] (1, -1) -- (1, -2);
  \draw (-2, -5) -- (2, -5) -- (2, -2) -- (-2, -2) -- (-2, -5);
  \coordinate[label=below:{\textbf{S}$_{2}$}] (s_2) at (0, -5);

    \node[o-point] (jo_2) [label=below:{Jo}] at (0, -4.25) {};
    \node[o-point] (n_1_2) [label=above:{$n_{1}$}] at (-1.25, -3) {};
    \node[o-point] (n_2_2) [label=above:{$n_{2}$}] at (0, -2.75) {};
    \node[o-point] (n_3_2) [label=above:{$n_{3}$}] at (1.25, -3) {};
  
    \draw[spaced-arrows,->] (jo_2) to node [fill=white] {holds} (n_3_2);

\end{diagram}

\noindent
Now suppose that if Jo is in state $S_{1}$ (i.e., if she's picked up the third nickel), then she can pick up the first nickel as well. Then she'll have two nickels in her hand:

\begin{diagram}

  % State 0
  \draw (-2, -1) -- (2, -1) -- (2, 2) -- (-2, 2) -- (-2, -1);
  \coordinate[label=below:{\textbf{S}$_{0}$}] (s_0) at (0, -1);
  
    \node[o-point] (jo) [label=below:{Jo}] at (0, -0.25) {};
    \node[o-point] (n_1) [label=above:{$n_{1}$}] at (-1.25, 1) {};
    \node[o-point] (n_2) [label=above:{$n_{2}$}] at (0, 1.25) {};
    \node[o-point] (n_3) [label=above:{$n_{3}$}] at (1.25, 1) {};
  
  % State 1
  \draw[spaced-arrows,->] (2, 0.5) -- (4, 0.5);
  \draw (4, -1) -- (8, -1) -- (8, 2) -- (4, 2) -- (4, -1);
  \coordinate[label=below:{\textbf{S}$_{1}$}] (s_1) at (6, -1);

    \node[o-point] (jo_1) [label=below:{Jo}] at (6, -0.25) {};
    \node[o-point] (n_1_1) [label=above:{$n_{1}$}] at (4.75, 1) {};
    \node[o-point] (n_2_1) [label=above:{$n_{2}$}] at (6, 1.25) {};
    \node[o-point] (n_3_1) [label=above:{$n_{3}$}] at (7.25, 1) {};
  
    \draw[spaced-arrows,->] (jo_1) to node [fill=white] {holds} (n_1_1);

  % State 2
  \draw[spaced-arrows,->] (1, -1) -- (1, -2);
  \draw (-2, -5) -- (2, -5) -- (2, -2) -- (-2, -2) -- (-2, -5);
  \coordinate[label=below:{\textbf{S}$_{2}$}] (s_2) at (0, -5);

    \node[o-point] (jo_2) [label=below:{Jo}] at (0, -4.25) {};
    \node[o-point] (n_1_2) [label=above:{$n_{1}$}] at (-1.25, -3) {};
    \node[o-point] (n_2_2) [label=above:{$n_{2}$}] at (0, -2.75) {};
    \node[o-point] (n_3_2) [label=above:{$n_{3}$}] at (1.25, -3) {};
  
    \draw[spaced-arrows,->] (jo_2) to node [fill=white] {holds} (n_3_2);

  % State 3
  \draw[spaced-arrows,->] (7, -1) -- (7, -2);
  \draw (4, -5) -- (8, -5) -- (8, -2) -- (4, -2) -- (4, -5);
  \coordinate[label=below:{\textbf{S}$_{3}$}] (s_3) at (6, -5);

    \node[o-point] (jo_3) [label=below:{Jo}] at (6, -4.25) {};
    \node[o-point] (n_1_3) [label=above:{$n_{1}$}] at (4.75, -3) {};
    \node[o-point] (n_2_3) [label=above:{$n_{2}$}] at (6, -2.75) {};
    \node[o-point] (n_3_3) [label=above:{$n_{3}$}] at (7.25, -3) {};

    \draw[spaced-arrows,->] (jo_3) to node [fill=white] {holds} (n_1_3);
    \draw[spaced-arrows,->] (jo_3) to node [fill=white] {holds} (n_3_3);

\end{diagram}

\noindent
Alternatively, let us suppose that if Jo is in state $S_{2}$ (i.e., she has nickel three in her hand), she can pick up the first nickel. Then she would have two nickels in her hand as well: the first and third nickel. This is exactly $S_{3}$, so we can draw an arrow from $S_{2}$ to $S_{3}$:

\begin{diagram}

  % State 0
  \draw (-2, -1) -- (2, -1) -- (2, 2) -- (-2, 2) -- (-2, -1);
  \coordinate[label=below:{\textbf{S}$_{0}$}] (s_0) at (0, -1);
  
    \node[o-point] (jo) [label=below:{Jo}] at (0, -0.25) {};
    \node[o-point] (n_1) [label=above:{$n_{1}$}] at (-1.25, 1) {};
    \node[o-point] (n_2) [label=above:{$n_{2}$}] at (0, 1.25) {};
    \node[o-point] (n_3) [label=above:{$n_{3}$}] at (1.25, 1) {};
  
  % State 1
  \draw[spaced-arrows,->] (2, 0.5) -- (4, 0.5);
  \draw (4, -1) -- (8, -1) -- (8, 2) -- (4, 2) -- (4, -1);
  \coordinate[label=below:{\textbf{S}$_{1}$}] (s_1) at (6, -1);

    \node[o-point] (jo_1) [label=below:{Jo}] at (6, -0.25) {};
    \node[o-point] (n_1_1) [label=above:{$n_{1}$}] at (4.75, 1) {};
    \node[o-point] (n_2_1) [label=above:{$n_{2}$}] at (6, 1.25) {};
    \node[o-point] (n_3_1) [label=above:{$n_{3}$}] at (7.25, 1) {};
  
    \draw[spaced-arrows,->] (jo_1) to node [fill=white] {holds} (n_1_1);

  % State 2
  \draw[spaced-arrows,->] (1, -1) -- (1, -2);
  \draw (-2, -5) -- (2, -5) -- (2, -2) -- (-2, -2) -- (-2, -5);
  \coordinate[label=below:{\textbf{S}$_{2}$}] (s_2) at (0, -5);

    \node[o-point] (jo_2) [label=below:{Jo}] at (0, -4.25) {};
    \node[o-point] (n_1_2) [label=above:{$n_{1}$}] at (-1.25, -3) {};
    \node[o-point] (n_2_2) [label=above:{$n_{2}$}] at (0, -2.75) {};
    \node[o-point] (n_3_2) [label=above:{$n_{3}$}] at (1.25, -3) {};
  
    \draw[spaced-arrows,->] (jo_2) to node [fill=white] {holds} (n_3_2);

  % State 3
  \draw[spaced-arrows,->] (2, -3.5) -- (4, -3.5);
  \draw[spaced-arrows,->] (7, -1) -- (7, -2);
  \draw (4, -5) -- (8, -5) -- (8, -2) -- (4, -2) -- (4, -5);
  \coordinate[label=below:{\textbf{S}$_{3}$}] (s_3) at (6, -5);

    \node[o-point] (jo_3) [label=below:{Jo}] at (6, -4.25) {};
    \node[o-point] (n_1_3) [label=above:{$n_{1}$}] at (4.75, -3) {};
    \node[o-point] (n_2_3) [label=above:{$n_{2}$}] at (6, -2.75) {};
    \node[o-point] (n_3_3) [label=above:{$n_{3}$}] at (7.25, -3) {};

    \draw[spaced-arrows,->] (jo_3) to node [fill=white] {holds} (n_1_3);
    \draw[spaced-arrows,->] (jo_3) to node [fill=white] {holds} (n_3_3);

\end{diagram}

\noindent
This diagram shows that Jo can start with no nickels, and she can end up holding both $n_{1}$ and $n_{3}$. To get there, she can take two paths: she can pick up $n_{1}$ first, or she can pick up $n_{3}$ first.


%%%%%%%%%%%%%%%%%%%%%%%%%%%%%%%%%%%%%%%%%
%%%%%%%%%%%%%%%%%%%%%%%%%%%%%%%%%%%%%%%%%
\section{Adding loops}

Suppose now that after Jo has picked up both nickels, she can drop them. To diagram this, we can draw an arrow from $S_{3}$ back to $S_{0}$:

\begin{diagram}

  % State 0
  \draw[spaced-arrows,->] (4, -2) -- (2, -1);
  \draw (-2, -1) -- (2, -1) -- (2, 2) -- (-2, 2) -- (-2, -1);
  \coordinate[label=below:{\textbf{S}$_{0}$}] (s_0) at (0, -1);
  
    \node[o-point] (jo) [label=below:{Jo}] at (0, -0.25) {};
    \node[o-point] (n_1) [label=above:{$n_{1}$}] at (-1.25, 1) {};
    \node[o-point] (n_2) [label=above:{$n_{2}$}] at (0, 1.25) {};
    \node[o-point] (n_3) [label=above:{$n_{3}$}] at (1.25, 1) {};
  
  % State 1
  \draw[spaced-arrows,->] (2, 0.5) -- (4, 0.5);
  \draw (4, -1) -- (8, -1) -- (8, 2) -- (4, 2) -- (4, -1);
  \coordinate[label=below:{\textbf{S}$_{1}$}] (s_1) at (6, -1);

    \node[o-point] (jo_1) [label=below:{Jo}] at (6, -0.25) {};
    \node[o-point] (n_1_1) [label=above:{$n_{1}$}] at (4.75, 1) {};
    \node[o-point] (n_2_1) [label=above:{$n_{2}$}] at (6, 1.25) {};
    \node[o-point] (n_3_1) [label=above:{$n_{3}$}] at (7.25, 1) {};
  
    \draw[spaced-arrows,->] (jo_1) to node [fill=white] {holds} (n_1_1);

  % State 2
  \draw[spaced-arrows,->] (1, -1) -- (1, -2);
  \draw (-2, -5) -- (2, -5) -- (2, -2) -- (-2, -2) -- (-2, -5);
  \coordinate[label=below:{\textbf{S}$_{2}$}] (s_2) at (0, -5);

    \node[o-point] (jo_2) [label=below:{Jo}] at (0, -4.25) {};
    \node[o-point] (n_1_2) [label=above:{$n_{1}$}] at (-1.25, -3) {};
    \node[o-point] (n_2_2) [label=above:{$n_{2}$}] at (0, -2.75) {};
    \node[o-point] (n_3_2) [label=above:{$n_{3}$}] at (1.25, -3) {};
  
    \draw[spaced-arrows,->] (jo_2) to node [fill=white] {holds} (n_3_2);

  % State 3
  \draw[spaced-arrows,->] (2, -3.5) -- (4, -3.5);
  \draw[spaced-arrows,->] (7, -1) -- (7, -2);
  \draw (4, -5) -- (8, -5) -- (8, -2) -- (4, -2) -- (4, -5);
  \coordinate[label=below:{\textbf{S}$_{3}$}] (s_3) at (6, -5);

    \node[o-point] (jo_3) [label=below:{Jo}] at (6, -4.25) {};
    \node[o-point] (n_1_3) [label=above:{$n_{1}$}] at (4.75, -3) {};
    \node[o-point] (n_2_3) [label=above:{$n_{2}$}] at (6, -2.75) {};
    \node[o-point] (n_3_3) [label=above:{$n_{3}$}] at (7.25, -3) {};

    \draw[spaced-arrows,->] (jo_3) to node [fill=white] {holds} (n_1_3);
    \draw[spaced-arrows,->] (jo_3) to node [fill=white] {holds} (n_3_3);

\end{diagram}

\noindent
We can add more loops. Suppose that after Jo picks up the first nickel, she can put it down (and so return to state $S_{0}$). To diagram that, we can draw an arrow from $S_{1}$ back to $S_{0}$.

\begin{diagram}

  % State 0
  \draw[spaced-arrows,->] (4, 0) -- (2, 0);
  \draw[spaced-arrows,->] (4, -2) -- (2, -1);
  \draw (-2, -1) -- (2, -1) -- (2, 2) -- (-2, 2) -- (-2, -1);
  \coordinate[label=below:{\textbf{S}$_{0}$}] (s_0) at (0, -1);
  
    \node[o-point] (jo) [label=below:{Jo}] at (0, -0.25) {};
    \node[o-point] (n_1) [label=above:{$n_{1}$}] at (-1.25, 1) {};
    \node[o-point] (n_2) [label=above:{$n_{2}$}] at (0, 1.25) {};
    \node[o-point] (n_3) [label=above:{$n_{3}$}] at (1.25, 1) {};
  
  % State 1
  \draw[spaced-arrows,->] (2, 0.5) -- (4, 0.5);
  \draw (4, -1) -- (8, -1) -- (8, 2) -- (4, 2) -- (4, -1);
  \coordinate[label=below:{\textbf{S}$_{1}$}] (s_1) at (6, -1);

    \node[o-point] (jo_1) [label=below:{Jo}] at (6, -0.25) {};
    \node[o-point] (n_1_1) [label=above:{$n_{1}$}] at (4.75, 1) {};
    \node[o-point] (n_2_1) [label=above:{$n_{2}$}] at (6, 1.25) {};
    \node[o-point] (n_3_1) [label=above:{$n_{3}$}] at (7.25, 1) {};
  
    \draw[spaced-arrows,->] (jo_1) to node [fill=white] {holds} (n_1_1);

  % State 2
  \draw[spaced-arrows,->] (1, -1) -- (1, -2);
  \draw (-2, -5) -- (2, -5) -- (2, -2) -- (-2, -2) -- (-2, -5);
  \coordinate[label=below:{\textbf{S}$_{2}$}] (s_2) at (0, -5);

    \node[o-point] (jo_2) [label=below:{Jo}] at (0, -4.25) {};
    \node[o-point] (n_1_2) [label=above:{$n_{1}$}] at (-1.25, -3) {};
    \node[o-point] (n_2_2) [label=above:{$n_{2}$}] at (0, -2.75) {};
    \node[o-point] (n_3_2) [label=above:{$n_{3}$}] at (1.25, -3) {};
  
    \draw[spaced-arrows,->] (jo_2) to node [fill=white] {holds} (n_3_2);

  % State 3
  \draw[spaced-arrows,->] (2, -3.5) -- (4, -3.5);
  \draw[spaced-arrows,->] (7, -1) -- (7, -2);
  \draw (4, -5) -- (8, -5) -- (8, -2) -- (4, -2) -- (4, -5);
  \coordinate[label=below:{\textbf{S}$_{3}$}] (s_3) at (6, -5);

    \node[o-point] (jo_3) [label=below:{Jo}] at (6, -4.25) {};
    \node[o-point] (n_1_3) [label=above:{$n_{1}$}] at (4.75, -3) {};
    \node[o-point] (n_2_3) [label=above:{$n_{2}$}] at (6, -2.75) {};
    \node[o-point] (n_3_3) [label=above:{$n_{3}$}] at (7.25, -3) {};

    \draw[spaced-arrows,->] (jo_3) to node [fill=white] {holds} (n_1_3);
    \draw[spaced-arrows,->] (jo_3) to node [fill=white] {holds} (n_3_3);

\end{diagram}

\noindent
Also, suppose that after Jo picks up the third nickel, she can put it down (and so return back to state $S_{0}$). To diagram that, we can add an arrow from $S_{2}$ back to $S_{0}$:

\begin{diagram}

  % State 0
  \draw[spaced-arrows,<-] (-1, -1) -- (-1, -2);
  \draw[spaced-arrows,->] (4, 0) -- (2, 0);
  \draw[spaced-arrows,->] (4, -2) -- (2, -1);
  \draw (-2, -1) -- (2, -1) -- (2, 2) -- (-2, 2) -- (-2, -1);
  \coordinate[label=below:{\textbf{S}$_{0}$}] (s_0) at (0, -1);
  
    \node[o-point] (jo) [label=below:{Jo}] at (0, -0.25) {};
    \node[o-point] (n_1) [label=above:{$n_{1}$}] at (-1.25, 1) {};
    \node[o-point] (n_2) [label=above:{$n_{2}$}] at (0, 1.25) {};
    \node[o-point] (n_3) [label=above:{$n_{3}$}] at (1.25, 1) {};
  
  % State 1
  \draw[spaced-arrows,->] (2, 0.5) -- (4, 0.5);
  \draw (4, -1) -- (8, -1) -- (8, 2) -- (4, 2) -- (4, -1);
  \coordinate[label=below:{\textbf{S}$_{1}$}] (s_1) at (6, -1);

    \node[o-point] (jo_1) [label=below:{Jo}] at (6, -0.25) {};
    \node[o-point] (n_1_1) [label=above:{$n_{1}$}] at (4.75, 1) {};
    \node[o-point] (n_2_1) [label=above:{$n_{2}$}] at (6, 1.25) {};
    \node[o-point] (n_3_1) [label=above:{$n_{3}$}] at (7.25, 1) {};
  
    \draw[spaced-arrows,->] (jo_1) to node [fill=white] {holds} (n_1_1);

  % State 2
  \draw[spaced-arrows,->] (1, -1) -- (1, -2);
  \draw (-2, -5) -- (2, -5) -- (2, -2) -- (-2, -2) -- (-2, -5);
  \coordinate[label=below:{\textbf{S}$_{2}$}] (s_2) at (0, -5);

    \node[o-point] (jo_2) [label=below:{Jo}] at (0, -4.25) {};
    \node[o-point] (n_1_2) [label=above:{$n_{1}$}] at (-1.25, -3) {};
    \node[o-point] (n_2_2) [label=above:{$n_{2}$}] at (0, -2.75) {};
    \node[o-point] (n_3_2) [label=above:{$n_{3}$}] at (1.25, -3) {};
  
    \draw[spaced-arrows,->] (jo_2) to node [fill=white] {holds} (n_3_2);

  % State 3
  \draw[spaced-arrows,->] (2, -3.5) -- (4, -3.5);
  \draw[spaced-arrows,->] (7, -1) -- (7, -2);
  \draw (4, -5) -- (8, -5) -- (8, -2) -- (4, -2) -- (4, -5);
  \coordinate[label=below:{\textbf{S}$_{3}$}] (s_3) at (6, -5);

    \node[o-point] (jo_3) [label=below:{Jo}] at (6, -4.25) {};
    \node[o-point] (n_1_3) [label=above:{$n_{1}$}] at (4.75, -3) {};
    \node[o-point] (n_2_3) [label=above:{$n_{2}$}] at (6, -2.75) {};
    \node[o-point] (n_3_3) [label=above:{$n_{3}$}] at (7.25, -3) {};

    \draw[spaced-arrows,->] (jo_3) to node [fill=white] {holds} (n_1_3);
    \draw[spaced-arrows,->] (jo_3) to node [fill=white] {holds} (n_3_3);

\end{diagram}

\noindent
Likewise, suppose that after Jo picks up both nickels, she can put one down (and so return to state $S_{1}$ or $_{2}$). We can add arrows to show that too:


\begin{diagram}

  % State 0
  \draw[spaced-arrows,<-] (-1, -1) -- (-1, -2);
  \draw[spaced-arrows,->] (4, 0) -- (2, 0);
  \draw[spaced-arrows,->] (4, -2) -- (2, -1);
  \draw (-2, -1) -- (2, -1) -- (2, 2) -- (-2, 2) -- (-2, -1);
  \coordinate[label=below:{\textbf{S}$_{0}$}] (s_0) at (0, -1);
  
    \node[o-point] (jo) [label=below:{Jo}] at (0, -0.25) {};
    \node[o-point] (n_1) [label=above:{$n_{1}$}] at (-1.25, 1) {};
    \node[o-point] (n_2) [label=above:{$n_{2}$}] at (0, 1.25) {};
    \node[o-point] (n_3) [label=above:{$n_{3}$}] at (1.25, 1) {};
  
  % State 1
  \draw[spaced-arrows,<-] (5, -1) -- (5, -2);
  \draw[spaced-arrows,->] (2, 0.5) -- (4, 0.5);
  \draw (4, -1) -- (8, -1) -- (8, 2) -- (4, 2) -- (4, -1);
  \coordinate[label=below:{\textbf{S}$_{1}$}] (s_1) at (6, -1);

    \node[o-point] (jo_1) [label=below:{Jo}] at (6, -0.25) {};
    \node[o-point] (n_1_1) [label=above:{$n_{1}$}] at (4.75, 1) {};
    \node[o-point] (n_2_1) [label=above:{$n_{2}$}] at (6, 1.25) {};
    \node[o-point] (n_3_1) [label=above:{$n_{3}$}] at (7.25, 1) {};
  
    \draw[spaced-arrows,->] (jo_1) to node [fill=white] {holds} (n_1_1);

  % State 2
  \draw[spaced-arrows,<-] (2, -4) -- (4, -4);
  \draw[spaced-arrows,->] (1, -1) -- (1, -2);
  \draw (-2, -5) -- (2, -5) -- (2, -2) -- (-2, -2) -- (-2, -5);
  \coordinate[label=below:{\textbf{S}$_{2}$}] (s_2) at (0, -5);

    \node[o-point] (jo_2) [label=below:{Jo}] at (0, -4.25) {};
    \node[o-point] (n_1_2) [label=above:{$n_{1}$}] at (-1.25, -3) {};
    \node[o-point] (n_2_2) [label=above:{$n_{2}$}] at (0, -2.75) {};
    \node[o-point] (n_3_2) [label=above:{$n_{3}$}] at (1.25, -3) {};
  
    \draw[spaced-arrows,->] (jo_2) to node [fill=white] {holds} (n_3_2);

  % State 3
  \draw[spaced-arrows,->] (2, -3.5) -- (4, -3.5);
  \draw[spaced-arrows,->] (7, -1) -- (7, -2);
  \draw (4, -5) -- (8, -5) -- (8, -2) -- (4, -2) -- (4, -5);
  \coordinate[label=below:{\textbf{S}$_{3}$}] (s_3) at (6, -5);

    \node[o-point] (jo_3) [label=below:{Jo}] at (6, -4.25) {};
    \node[o-point] (n_1_3) [label=above:{$n_{1}$}] at (4.75, -3) {};
    \node[o-point] (n_2_3) [label=above:{$n_{2}$}] at (6, -2.75) {};
    \node[o-point] (n_3_3) [label=above:{$n_{3}$}] at (7.25, -3) {};

    \draw[spaced-arrows,->] (jo_3) to node [fill=white] {holds} (n_1_3);
    \draw[spaced-arrows,->] (jo_3) to node [fill=white] {holds} (n_3_3);

\end{diagram}


%%%%%%%%%%%%%%%%%%%%%%%%%%%%%%%%%%%%%%%%%
%%%%%%%%%%%%%%%%%%%%%%%%%%%%%%%%%%%%%%%%%
\section{Summary}

Transition systems can be modeled with any level of complexity. Even in a simple system like we have modeled here, loops and branches can depict various complexities of the system.


\end{document}
