\documentclass[../../../main.tex]{subfiles}
\begin{document}

%%%%%%%%%%%%%%%%%%%%%%%%%%%%%%%%%%%%%%%%%
%%%%%%%%%%%%%%%%%%%%%%%%%%%%%%%%%%%%%%%%%
%%%%%%%%%%%%%%%%%%%%%%%%%%%%%%%%%%%%%%%%%
\chapter{A first proof}


%%%%%%%%%%%%%%%%%%%%%%%%%%%%%%%%%%%%%%%%%
%%%%%%%%%%%%%%%%%%%%%%%%%%%%%%%%%%%%%%%%%
\section{Getting cucumbers}

Suppose we want to prove that I can get cucumbers in my basket, under the assumptions that I start with potatoes in my basket and I can trade potatoes for cucumbers. Let's write that out as a hypothetical judgment:

\begin{equation*}
  b(p) :: true, t(p, c) :: true \ndturnstile/ b(c) :: true
\end{equation*}

\noindent
We can build a proof tree of this judgment in one of two ways. We can start at the top of the tree and work down, or we can start at the bottom of the tree and work up.


%%%%%%%%%%%%%%%%%%%%%%%%%%%%%%%%%%%%%%%%%
%%%%%%%%%%%%%%%%%%%%%%%%%%%%%%%%%%%%%%%%%
\section{Work top-down}

To work top down means we start with the assumptions, and try to work downwards until we get to the conclusion.

So, let's pretend we're in a scenario where the first assumption is true. That is, let's pretend we're in a scenario where I do start with potatoes in my basket. We can use the \rulename{start} rule to assert this:

\begin{prooftree*}
  \hypo{}
  \infer1[\rulename{start}]{b(p) :: true}
\end{prooftree*}

\noindent
Let's also pretend that the second assumption is correct --- I can trade potatoes for cucumbers. We can use the \rulename{start} rule to assert this as well, next to the first assumption:

\begin{prooftree*}
  \hypo{}
  \infer1[\rulename{start}]{b(p) :: true}
  
  \hypo{}
  \infer1[\rulename{start}]{t(p, c) :: true}
  
  \infer[rule style=no rule]2{}
\end{prooftree*}

\noindent
So now we're in a scenario where I start with potatoes in my basket, and where I can trade potatoes for cucumbers. This matches the assumptions in the hypothetical judgment above. If we were modeling this, this would be the start-state $S_{0}$.

What's the next step in the proof? We are working downwards, so we want to fill in what comes below these premises, to get to the conclusion. Like this:

\begin{prooftree*}
  \hypo{}
  \infer1[\rulename{start}]{b(p) :: true}
  \ellipsis{}{}
  \infer[rule style=no rule]1{??}

  \hypo{}
  \infer1[\rulename{start}]{t(p, c) :: true}

  \ellipsis{}{}
  \infer[rule style=no rule]1{??}
  
  \infer[rule style=no rule]2{}

\end{prooftree*}

\noindent
To figure out what comes next, we have to figure out what rule we can apply to derive a further conclusion. 

Fortunately, we can use the \rulename{trade} rule, because we have $b(p) :: true$ and $t(p, c) :: true$, and those match the premises of the \rulename{trade} rule.

If we write the \rulename{trade} rule template below these assumptions, we can see that it fits right up into what we have:

\begin{prooftree*}
  \hypo{}
  \infer1[\rulename{start}]{b(p) :: true}
  \infer[rule style=no rule]1{\left\uparrow\rule{0cm}{0.75cm}\right.}

  \infer[rule style=no rule]1{b(x) :: true}

  \hypo{}
  \infer1[\rulename{start}]{t(p, c) :: true}

  \infer[rule style=no rule]1{\left\uparrow\rule{0cm}{0.75cm}\right.}

  \infer[rule style=no rule]1{t(x, y) :: true}
  \infer2[\rulename{trade}]{b(y) :: true}

\end{prooftree*}

\noindent
But we need to replace the $x$s and $y$s in the template with the $p$s and $c$s from our assumptions. Once we do that replacement, it looks like this:

\begin{prooftree*}
  \hypo{}
  \infer1[\rulename{start}]{b(p) :: true}
  \infer[rule style=no rule]1{\left\uparrow\rule{0cm}{0.75cm}\right.}

  \infer[rule style=no rule]1{b(p) :: true}

  \hypo{}
  \infer1[\rulename{start}]{t(p, c) :: true}

  \infer[rule style=no rule]1{\left\uparrow\rule{0cm}{0.75cm}\right.}

  \infer[rule style=no rule]1{t(p, c) :: true}
  \infer2[\rulename{trade}]{b(c) :: true}

\end{prooftree*}

\noindent
Now we can see the perfect fit. So, we can collapse the arrows, to get this:

\begin{prooftree*}
  \hypo{}
  \infer1[\rulename{start}]{b(p) :: true}
  \hypo{}
  \infer1[\rulename{start}]{t(p, c) :: true}
  \infer2[\rulename{trade}]{b(c) :: true}
\end{prooftree*}

\noindent
That completes the proof. We've reached the conclusion we wanted to reach, namely ``$b(c) :: true$,'' which is the conclusion of the hypothetical judgment we started with. 


%%%%%%%%%%%%%%%%%%%%%%%%%%%%%%%%%%%%%%%%%
%%%%%%%%%%%%%%%%%%%%%%%%%%%%%%%%%%%%%%%%%
\section{Commentary}

Let's write out each step of this proof tree, so we can see the sequence of moves. I will write a number next to each judgment, and I will write a justification comment to the right of each judgment, to explain what happens in that step. Here it is:

\begin{center}
\begin{tabular}{l  l  l}
  (1) & $b(p) :: true$ & Assume this holds initially, by the \rulename{start} rule. \\
  (2) & $t(p, c) :: true$ & Assume this holds initially, by the \rulename{start} rule. \\
  (3) & $b(c) :: true$ & Infer this from (1) and (2), by the \rulename{trade} rule.
\end{tabular}
\end{center}

\noindent
Here is some more commentary on each step of that proof:

\begin{itemize}
  \item{We can see that in step (1), we use the \rulename{start} rule to assume from the start that $b(p)$ is true (i.e., we assume that I have potatoes in my basket).}
  \item{Then, in step (2), we use the \rulename{start} rule again, to assume from the start that $t(p, c)$ is true (i.e., we assume that I can trade potatoes for cucumbers).}
  \item{In the last step (3), we use the \rulename{trade} rule, where the judgments from steps (1) and (2) are the premises. And that yields the conclusion, that $b(c)$ is true (i.e., that cucumbers end up in my basket).}
\end{itemize}


%%%%%%%%%%%%%%%%%%%%%%%%%%%%%%%%%%%%%%%%%
%%%%%%%%%%%%%%%%%%%%%%%%%%%%%%%%%%%%%%%%%
\section{Work bottom-up}

We can also work bottom-up, and this is in fact often easier. To do this, we start by writing the conclusion we want to prove first:

\begin{prooftree*}
  \hypo{}
  \hypo{}
  \infer[rule style=no rule]2{b(c) :: true}
\end{prooftree*}

\noindent
Then we want to figure out what goes above it. To do that, we ask: what rule would we have to apply, to get this as the conclusion? What would have to be above the line, to get this?

\begin{prooftree*}
  \hypo{??}
  \ellipsis{}{}
  \infer1[]{b(c) :: true}
\end{prooftree*}

\noindent
In this case, it's clear that we have to use the \rulename{trade} rule to get this conclusion. Why? It can't be the \rulename{start} rule, because $b(c) :: true$ is not one of the assumptions in the hypothetical judgment we're trying to prove. The only other rule is the \rulename{trade} rule, so it must be that.

Now that we know which rule must be applied, we can fill it in. So we write \rulename{trade} next to the inference line:

\begin{prooftree*}
  \hypo{}
  \hypo{}
  \infer2[trade]{b(c) :: true}
\end{prooftree*}

\noindent
Then we need to fill in the premises. The \rulename{trade} rule has two premises, so we know we have to fill in two premises above the line:

\begin{prooftree*}
  \hypo{\fbox{??}}
  \hypo{\fbox{??}}
  \infer2[trade]{b(c) :: true}
\end{prooftree*}

\noindent
What are the premises? Well, we can start by filling in what the rule's template has for its premises:

\begin{prooftree*}
  \hypo{\fbox{$b(x) :: true$}}
  \hypo{\fbox{$t(x, y) :: true$}}
  \infer2[trade]{b(c) :: true}
\end{prooftree*}

\noindent
However, the template's premises have $x$s and $y$s in them. We need to replace the $x$s and $y$s with real values. We know what $y$ must be, since we know that it's $c$ from the conclusion. So put that in:

\begin{prooftree*}
  \hypo{b(x) :: true}
  \hypo{t(x, \fbox{c}) :: true}
  \infer2[trade]{b(\fbox{c}) :: true}
\end{prooftree*}

\noindent
Now we need to fill in $x$. To do this, we have to look at the assumptions of the judgment we're trying to prove, to see if we can get some help. In particular, we want to find an assumption that fits $t(x, c)$. 

We have two assumptions in the hypothetical judgment we're trying to prove, namely ``$b(p) :: true$'' and ``$t(p, c) :: true$.'' And indeed, one of them matches: ``$t(p, c) :: true$'' matches the $c$. So we know that we should replace $x$ with $p$:

\begin{prooftree*}
  \hypo{b(\fbox{p}) :: true}
  \hypo{t(\fbox{p}, c) :: true}
  \infer2[trade]{b(c) :: true}
\end{prooftree*}

\noindent
Now we have completed the \rulename{trade} rule. We started with the conclusion, and then we worked upwards one step, to fill in the premises that must have been used to derive the conclusion.

Next we turn to \emph{those} premises, and repeat the process. For each of those premises, what rule would we have to apply to derive them? 

Look at the first premise: $b(p) :: true$. Is there a rule that could derive this? We only have two rules: the \rulename{start} rule, or the \rulename{trade} rule. The \rulename{start} rule works here, because $b(p) :: true$ is an assumption in the hypothetical judgment we're trying to prove, and the \rulename{start} rule is precisely the rule that lets us assert an assumption. So, we can draw a line above $b(p) :: true$ and write \rulename{start} next to it:

\begin{prooftree*}
  \hypo{}
  \infer1[\rulename{start}]{b(p) :: true}
  \hypo{t(p, c) :: true}
  \infer2[trade]{b(c) :: true}
\end{prooftree*}

\noindent
Now look at the top right premise: $t(p, c) :: true$. Is there a rule that could derive this? Again, we only have two rules, and the \rulename{start} rule works. So we can draw a line above $t(p, c) :: true$ and write \rulename{start} next to it:

\begin{prooftree*}
  \hypo{}
  \infer1[\rulename{start}]{b(p) :: true}
  \hypo{}
  \infer1[\rulename{start}]{t(p, c) :: true}
  \infer2[trade]{b(c) :: true}
\end{prooftree*}

\noindent
At this point, every branch at the top of the tree is closed off by a \rulename{start} rule, so we cannot go any further. Our proof tree is complete.


%%%%%%%%%%%%%%%%%%%%%%%%%%%%%%%%%%%%%%%%%
%%%%%%%%%%%%%%%%%%%%%%%%%%%%%%%%%%%%%%%%%
\section{Proofs of hypothetical judgments}

Whether we start from the assumptions and work downwards to the conclusion, or whether we start with the conclusion and work upwards to the assumptions, we end up constructing the same proof tree. And this proof tree is a proof of the hypothetical judgment we originally set out to prove.

The point to notice here is that the proof does indeed match the hypothetical judgment we set out to prove. That hypothetical judgment was this:

\begin{equation*}
  b(p) :: true, t(p, c) :: true \ndturnstile/ b(c) :: true
\end{equation*}

\noindent
In this hypothetical judgment, we have two assumptions --- ``$b(p) :: true$'' and ``$t(p, c) :: true$.'' We also have a conclusion --- ``$b(c) :: true$.'' Notice where each of these pieces are positioned in the hypothetical judgment:

\begin{diagram}

  \node (j) [label=above:{$b(p) :: true, ~~~~ t(p, c) :: true ~~~ \ndturnstile/ ~~~ b(c) :: true$}] at (0, 0) {}; 

  \draw (1.75, 0.25) -- (1.75, 0) -- (3.55, 0) -- (3.55, 0.25);
  \draw[spaced-arrows,->] (3, -0.75) -- (2.6, 0);
  \node (c_1_l) [label=below:{conclusion}] at (3, -0.6) {};

  \draw (-1.35, 0.25) -- (-1.35, 0) -- (0.8, 0) -- (0.8, 0.25);
  \draw[spaced-arrows,->] (0.25, -1) -- (0, 0);
  \node (a_2_l) [label=below:{assumption 2}] at (0.25, -1) {}; 

  \draw (-3.65, 0.25) -- (-3.65, 0) -- (-1.6, 0) -- (-1.6, 0.25);
  \draw[spaced-arrows,->] (-2.5, -0.75) -- (-2.35, 0);
  \node (a_1_l) [label=below:{assumption 1}] at (-2.5, -0.75) {};

\end{diagram}

\noindent
The assumptions are each listed to the left of the turnstile, and the conclusion is to the right of the turnstile. 

Now look at where these assumptions and the conclusion are positioned on the proof tree:

\begin{diagram}

  \node (j) [] at (0, 0) {
    \begin{prooftree}
      \hypo{}
      \infer1[\rulename{start}]{b(p) :: true}
      \hypo{}
      \infer1[\rulename{start}]{t(p, c) :: true}
      \infer2[trade]{b(c) :: true}
    \end{prooftree}
  };

  \draw (-1.5, -0.5) -- (-1.5, -0.75) -- (0.5, -0.75) -- (0.5, -0.5);
  \draw[spaced-arrows,->] (-0.5, -1.5) -- (-0.5, -0.75);
  \node (c_1_l) [label=below:{conclusion}] at (-0.5, -1.3) {};

  \draw (-3, 0.75) -- (-3, 1) -- (-0.5, 1) -- (-0.5, 0.75);
  \draw[spaced-arrows,->] (-2, 2) -- (-2, 1);
  \node (a_1_l) [label=above:{assumption 1}] at (-2, 1.9) {};

  \draw (-0.15, 0.75) -- (-0.15, 1) -- (2.6, 1) -- (2.6, 0.75);
  \draw[spaced-arrows,->] (1.25, 2) -- (1.25, 1);
  \node (a_2_l) [label=above:{assumption 2}] at (1.25, 1.9) {};
\end{diagram}

\noindent
The assumptions are all at the top, and the conclusion is at the bottom. The proof fills out everything in between.

Every proof matches a hypothetical judgment like this. For each assumption in a hypothetical judgment, there is a \rulename{start} for that assumption somewhere at the top of the tree. And of course, for the conclusion of the hypothetical judgment, we can always see it at the bottom of the proof tree.


%%%%%%%%%%%%%%%%%%%%%%%%%%%%%%%%%%%%%%%%%
%%%%%%%%%%%%%%%%%%%%%%%%%%%%%%%%%%%%%%%%%
\section{Summary}

To prove a hypothetical judgment in the linear natural deduction style, we can either work top-down, or bottom-up. 

To work from the top down, we first introduce each assumption with the \rulename{start} rule. Then we look at the assumptions available and apply any inference rules that have matching premises. Then we repeat this process until we reach the conclusion. 

When we work from the bottom up, we first put down the conclusion, then we reason backwards, and figure out which rule must have been applied to get that conclusion. Then we move up and do the same thing for the premises of that. We repeat the same process until we reach the initial assumptions, which we mark with the \rulename{start} rule.


\end{document}
