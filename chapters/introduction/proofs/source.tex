\documentclass[../../../main.tex]{subfiles}
\begin{document}

%%%%%%%%%%%%%%%%%%%%%%%%%%%%%%%%%%%%%%%%%
%%%%%%%%%%%%%%%%%%%%%%%%%%%%%%%%%%%%%%%%%
%%%%%%%%%%%%%%%%%%%%%%%%%%%%%%%%%%%%%%%%%
\chapter{Proofs}


%%%%%%%%%%%%%%%%%%%%%%%%%%%%%%%%%%%%%%%%%
%%%%%%%%%%%%%%%%%%%%%%%%%%%%%%%%%%%%%%%%%
\section{Proof systems}

A proof system is a collection of inference rules. We construct proofs by chaining together inference rules. 

So a proof is just an argument that has been constructed solely from inference rules. 


%%%%%%%%%%%%%%%%%%%%%%%%%%%%%%%%%%%%%%%%%
%%%%%%%%%%%%%%%%%%%%%%%%%%%%%%%%%%%%%%%%%
\section{Stacking inferences}

Suppose we have two inference rules that look something like this:

$$
  \begin{prooftree}
    \hypo{~~~~A~~~~}
    \infer1[\rulename{Tick}]{~~~~B~~~~}
  \end{prooftree}
  \hskip 2cm
  \begin{prooftree}
    \hypo{~~~~B~~~~}
    \infer1[\rulename{Tock}]{~~~~C~~~~}
  \end{prooftree}
$$

\noindent
These are just made up rules of course. \rulename{Tick} says if we have a statement of the form $A$, we can infer a statement of form $B$. \rulename{Tock} says if we have a statement of form $B$, we can infer a statement of form $C$.

Since the conclusion of \rulename{Tick} matches the premise of \rulename{Tock}, we can chain them together:

\begin{prooftree*}
  \hypo{~~~~A~~~~}
  \infer1[\rulename{Tick}]{~~~~B~~~~}
  \infer1[\rulename{Tock}]{~~~~C~~~~}
\end{prooftree*}

\noindent
We can visualize this chain of inferences another way:

\begin{diagram}

  \node (a) [] {$A$};
  
  \node (b) [below=of a] {$B$};
  
  \node (c) [below=of b] {$C$};

  \path[->] (a) edge [right] node {\rulename{Tick}} (b);
  \path[->] (b) edge [right] node {\rulename{Tock}} (c);
  
\end{diagram}


%%%%%%%%%%%%%%%%%%%%%%%%%%%%%%%%%%%%%%%%%
%%%%%%%%%%%%%%%%%%%%%%%%%%%%%%%%%%%%%%%%%
\section{Trees}

Suppose I have three rules that look something like this:

$$
  \begin{prooftree}
    \hypo{~~~~C~~~~}
    \infer1[\rulename{Fee}]{~~~~D~~~~}
  \end{prooftree}
  \hskip 1cm
  \begin{prooftree}
    \hypo{~~~~E~~~~}
    \infer1[\rulename{Fie}]{~~~~F~~~~}
  \end{prooftree}
  \hskip 1cm
  \begin{prooftree}
    \hypo{~~D}
    \hypo{F~~}
    \infer2[\rulename{Foe}]{G}
  \end{prooftree}  
$$

\noindent
Note that \rulename{Foe} has two premises above the line. \rulename{Fee} and \rulename{Fie} match those premises, so we can chain them together:

\begin{prooftree*}
  \hypo{~~~~C~~~~}
  \infer1[\rulename{Fee}]{~~D~~}

  \hypo{~~~~E~~~~}  
  \infer1[\rulename{Fie}]{~~F~~}

  \infer2[\rulename{Foe}]{G}
\end{prooftree*}

\noindent
This is a tree. The root is at the bottom, and there are two branches above it. To visualize it differently:

\begin{diagram}

  \node (g) [] {$G$};

  \node (d) [above left=of g] {$D$};
  \node (c) [above=of d] {$C$};
  
  \node (f) [above right=of g] {$F$};
  \node (e) [above=of f] {$E$};

  \path[->] (c) edge [left] node {\rulename{Fee}} (d);
  \path[->] (e) edge [right] node {\rulename{Fie}} (f);

  \path[->] (d) edge [left] node {\rulename{Foe~(left)}~~} (g);
  \path[->] (f) edge [right] node {~~\rulename{Foe~(right)}} (g);
  
\end{diagram}


%%%%%%%%%%%%%%%%%%%%%%%%%%%%%%%%%%%%%%%%%
%%%%%%%%%%%%%%%%%%%%%%%%%%%%%%%%%%%%%%%%%
\section{More branching}

Suppose I have two branching rules that look something like this:

$$
  \begin{prooftree}
    \hypo{~~D}
    \hypo{F~~}
    \infer2[\rulename{Tweedledee}]{G}
  \end{prooftree}
  \hskip 1cm
  \begin{prooftree}
    \hypo{~~G}
    \hypo{H~~}
    \infer2[\rulename{Tweedledum}]{J}
  \end{prooftree}  
$$

\noindent
The conclusion of \rulename{Tweedledee} matches the left premise of \rulename{Tweedledum}, so we can chain the two rules at that point:

\begin{prooftree*}

  \hypo{~~D}
  \hypo{F~~}
  \infer2[\rulename{Tweedledee}]{G}

  \hypo{H~~}
  \infer2[\rulename{Tweedledum}]{J}

\end{prooftree*}

\noindent
This tree has two branching points. Visually:

\begin{diagram}

  \node (j) [] {$J$};

  \node (g) [above left=of j] {$G$};
  \node (d) [above left=of g] {$D$};
  \node (f) [above right=of g] {$F$};

  \node (h) [above right=of j] {$H$};

  \path[->] (d) edge [left] node {\rulename{Tweedledee (left)~~}} (g);
  \path[->] (f) edge [right] node {~~\rulename{Tweedledee (right)}} (g);

  \path[->] (g) edge [left] node {\rulename{Tweedledum~(left)}~~} (j);
  \path[->] (h) edge [right] node {~~\rulename{Tweedledum~(right)}} (j);
  
\end{diagram}


%%%%%%%%%%%%%%%%%%%%%%%%%%%%%%%%%%%%%%%%%
%%%%%%%%%%%%%%%%%%%%%%%%%%%%%%%%%%%%%%%%%
\section{Summary}

A proof system is a collection of rules. A proof (in a proof system) is an argument constructed only with inference rules from that proof system. 

To construct proofs, we chain together inference rules. This generates a tree-like structure.


\end{document}
