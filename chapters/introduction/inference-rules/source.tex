\documentclass[../../../main.tex]{subfiles}
\begin{document}

%%%%%%%%%%%%%%%%%%%%%%%%%%%%%%%%%%%%%%%%%
%%%%%%%%%%%%%%%%%%%%%%%%%%%%%%%%%%%%%%%%%
%%%%%%%%%%%%%%%%%%%%%%%%%%%%%%%%%%%%%%%%%
\chapter{Inference rules}


%%%%%%%%%%%%%%%%%%%%%%%%%%%%%%%%%%%%%%%%%
%%%%%%%%%%%%%%%%%%%%%%%%%%%%%%%%%%%%%%%%%
\section{Templates}

Consider this argument:

\begin{prooftree*}
  \hypo{\text{All humans are animals.}}
  \hypo{\text{All animals are mortal.}}
  \infer2{\text{Therefore, all humans are mortal.}}
\end{prooftree*}

\noindent
It has a specific form:

\begin{prooftree*}
  \hypo{\text{All $A$s are $B$.}}
  \hypo{\text{All $B$s are $C$.}}
  \infer2{\text{Therefore, all $A$s are $C$.}}
\end{prooftree*}

\noindent
This is a \vocab{template} (synonym: \vocab{schema}) for an argument, in the sense that the italicized letters $A$, $B$, and $C$ are placeholders or holes that can be filled in by common terms like ``human,'' ``animal,'' and ``mortal.''


%%%%%%%%%%%%%%%%%%%%%%%%%%%%%%%%%%%%%%%%%
%%%%%%%%%%%%%%%%%%%%%%%%%%%%%%%%%%%%%%%%%
\section{Inference rules}

We can write down inference rules with templates. To do that, first draw a line, and write the name of the rule beside it, like this:

\begin{prooftree*}
  \hypo{~~~~~~~~~~~~~~~~~~~~~~~~~~~~~~}
  \infer1[$\mathsf{Barbara}$]{}
\end{prooftree*}

\noindent
Next, write down templates for all the premises above the line, side by side. For example:

\begin{prooftree*}
  \hypo{\text{All $A$s are $B$.}}
  \hypo{\text{All $B$s are $C$.}}
  \infer2[$\mathsf{Barbara}$]{}
\end{prooftree*}

\noindent
Finally, write down templates for the conclusions below the line:

\begin{prooftree*}
  \hypo{\text{All $A$s are $B$.}}
  \hypo{\text{All $B$s are $C$.}}
  \infer2[$\mathsf{Barbara}$]{\text{All $A$s are $C$.}}
\end{prooftree*}

\noindent
This is in fact a well known inference rule constructed long ago by Aristotle. The name ``Barbara'' was invented in the Middle Ages.


%%%%%%%%%%%%%%%%%%%%%%%%%%%%%%%%%%%%%%%%%
%%%%%%%%%%%%%%%%%%%%%%%%%%%%%%%%%%%%%%%%%
\section{Substitution instances}

The \rulename{Barbara} rule is a template. To construct a concrete argument, replace the placeholders with real values. We call the result a \vocab{substitution instance}.

For instance, we can replace every $A$ with ``organism,'' every $B$ with ``living thing,'' and every $C$ with ``physical being'':

\begin{prooftree*}
  \hypo{\text{All organisms are living things.}}
  \hypo{\text{All living things are physical things.}}
  \infer2[$\mathsf{Barbara}$]{\text{All organisms are physical things.}}
\end{prooftree*}

\noindent
This says that we used the \rulename{Barbara} rule to infer that ``All organisms are physical things'' from the premises ``All organisms are living things'' and ``All living things are physical things.'' 

Here is a different substitution instance:

\begin{prooftree*}
  \hypo{\text{All squares are 4-sided shapes.}}
  \hypo{\text{All 4-sided shapes are quadrangles.}}
  \infer2[$\mathsf{Barbara}$]{\text{All squares are quadrangles.}}
\end{prooftree*} 

\noindent
This says that we again used the \rulename{Barbara} rule to infer that ``All squares are quadrangles'' from the premises ``All squares are 4-sided shapes'' and ``All 4-sided shapes are quadrangles.'' 

Note that we must replace \textbf{every} $A$ with the same value. We cannot replace one occurrence of $A$ with ``squares,'' and then replace another occurrence with ``squirrels.'' We must substitute concrete values for the placeholders \vocab{uniformly}.


%%%%%%%%%%%%%%%%%%%%%%%%%%%%%%%%%%%%%%%%%
%%%%%%%%%%%%%%%%%%%%%%%%%%%%%%%%%%%%%%%%%
\section{Zero-premise rules}

In \rulename{Barbara}, there are two premise templates above the line, but inference rules can have zero or more. 

Here is a rule with no premises. Let $n$ be a placeholder for any number:

\begin{prooftree*}
  \hypo{}
  \infer1[$\mathsf{Numeric~identity}$]{n = n}
\end{prooftree*}

\noindent
This rule is called \rulename{Numeric~identity}, and it says that we can conclude that any number $n$ is the same as itself (we need no premises to draw this conclusion).

To use this rule, we can replace $n$ with any number. For example, fill in $n$ with $467$:

\begin{prooftree*}
  \hypo{}
  \infer1[$\mathsf{Numeric~identity}$]{467 = 467}
\end{prooftree*}

\noindent
This says we used the rule \rulename{Numeric~identity} to infer that ``$467 = 467$.'' 


%%%%%%%%%%%%%%%%%%%%%%%%%%%%%%%%%%%%%%%%%
%%%%%%%%%%%%%%%%%%%%%%%%%%%%%%%%%%%%%%%%%
\section{One-premise rules}

Here is a rule with one premise and one conclusion. Let $n$ and $m$ be placeholders for numbers again:

\begin{prooftree*}
  \hypo{n = m}
  \infer1[$\mathsf{Symmetry}$]{m = n}
\end{prooftree*}

\noindent
This rule says that if we have a statement of the form $n = m$, we can flip it around: $m = n$.

Here is a substitution instance of using the rule:

\begin{prooftree*}
  \hypo{4 = (2 * 2)}
  \infer1[$\mathsf{Symmetry}$]{(2 * 2) = 4}
\end{prooftree*}

\noindent
This says that we used the rule \rulename{Symmetry} to infer that ``$(2 *2) = 4$'' from the premise that ``$4 = (2 * 2)$.''


%%%%%%%%%%%%%%%%%%%%%%%%%%%%%%%%%%%%%%%%%
%%%%%%%%%%%%%%%%%%%%%%%%%%%%%%%%%%%%%%%%%
\section{Two or more-premise rules}

Here is a rule with three premises, and one conclusion.

\begin{prooftree*}
  \hypo{a = b}
  \hypo{b = c}
  \hypo{c = d}
  \infer3[$\mathsf{Three~links}$]{a = d}
\end{prooftree*}


%%%%%%%%%%%%%%%%%%%%%%%%%%%%%%%%%%%%%%%%%
%%%%%%%%%%%%%%%%%%%%%%%%%%%%%%%%%%%%%%%%%
\section{The general form}

We can have any number of premise templates above the line, and we can have any number of premise templates below the line. In general, an inference rule has this shape ---

\begin{prooftree*}
  \hypo{S_{1}}
  \hypo{S_{2}}
  \hypo{\ldots}
  \hypo{S_{n}}
  \infer4[$\mathsf{Name}$]{T_{1}, T_{2}, \ldots, T_{m}}
\end{prooftree*}

\noindent
--- where there can be any number of $S$s above the line, and any number of $T$s below the line.


%%%%%%%%%%%%%%%%%%%%%%%%%%%%%%%%%%%%%%%%%
%%%%%%%%%%%%%%%%%%%%%%%%%%%%%%%%%%%%%%%%%
\section{Summary}

We write down inference rules by drawing a line and writing a name next to it. We place templates for the required premises above the line, and we place templates for the conclusions below the line.

We can use an inference rule by uniformly replacing its placeholders with concrete values.

\end{document}
