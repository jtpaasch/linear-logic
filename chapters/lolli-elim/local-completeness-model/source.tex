\documentclass[../../../main.tex]{subfiles}
\begin{document}

%%%%%%%%%%%%%%%%%%%%%%%%%%%%%%%%%%%%%%%%%
%%%%%%%%%%%%%%%%%%%%%%%%%%%%%%%%%%%%%%%%%
%%%%%%%%%%%%%%%%%%%%%%%%%%%%%%%%%%%%%%%%%
\chapter{Modeling local completeness}


%%%%%%%%%%%%%%%%%%%%%%%%%%%%%%%%%%%%%%%%%
%%%%%%%%%%%%%%%%%%%%%%%%%%%%%%%%%%%%%%%%%
\section{The expansion}

Recall that, in order to prove that the lolli elimination rule is locally complete, we needed to show that if we eliminate a lolli, we can re-introduce it from the pieces that are left. 

We summarized the whole expansion like this:

$$
\begin{prooftree}
  \hypo{\Proof/}
  \ellipsis{}{}
  \infer[rule style=no rule]1{A \lolli/ B}
\end{prooftree}
\hskip 1cm\rightsquigarrow_{\eta}\hskip 1cm
\begin{prooftree}
  \hypo{\Proof/}
  \ellipsis{}{}
  \infer[rule style=no rule]1{A \lolli/ B}
  \hypo{}
  \infer1[\startrule/]{~~A^{a}~~}
  \infer2[\lolliElim/]{B}
  \infer1[\lolliIntro/$^{a}$]{A \lolli/ B}
\end{prooftree}
$$

\noindent
We can model this too, and the model offers us another way to see that the expansion makes sense.


%%%%%%%%%%%%%%%%%%%%%%%%%%%%%%%%%%%%%%%%%
%%%%%%%%%%%%%%%%%%%%%%%%%%%%%%%%%%%%%%%%%
\section{Model the lolli introduction}

First, we will model the lolli the expansion begins with, which is the part highlighted here:

\begin{diagram}

  \draw[densely dotted, fill=grey90] 
      (-1.9, 1.25) -- (-0.5, 1.25) -- (-0.5, -0.35) -- (-1.9, -0.35) -- (-1.9, 1.25);
  
  \node (j) [] at (0, 0) {
    \begin{prooftree}
      \hypo{\Proof/}
      \ellipsis{}{}
      \infer[rule style=no rule]1{A \lolli/ B}
      \hypo{}
      \infer1[\startrule/]{~~A^{a}~~}
      \infer2[\lolliElim/]{B}
      \infer1[\lolliIntro/$^{a}$]{A \lolli/ B}
    \end{prooftree}
  };

\end{diagram}

\noindent
Whatever the ``$\Proof/$'' fragment amounts to, we end up with a lolli from ``$A$'' to ``$B$.'' The model for that looks like this:

\begin{diagram}

  % State 0
  \draw[densely dotted] (-1, -0.75) -- (1.25, -0.75) -- (1.25, 1.75) -- (-1, 1.75) -- (-1, -0.75);
  \coordinate[label=below:{\textbf{S}$_{0}$}] (s_0) at (0.175, -0.75);

    \node[] (a) [] at (0.15, 0.5) {$A$};

  % State 1
  \draw[densely dotted,spaced-arrows,->] (1.25, 0.5) -- (2.25, 0.5);
  \draw[densely dotted] (2.25, -0.75) -- (4.5, -0.75) -- (4.5, 1.75) -- (2.25, 1.75) -- (2.25, -0.75);
  \coordinate[label=below:{\textbf{S}$_{1}$}] (s_1) at (3.5, -0.75);

    \node[] (b) [] at (3.35, 0.5) {$B$};

\end{diagram}

\noindent
This transition system shows that if we have a state where ``$A$'' is true, we can get to a state where ``$B$'' is true. This is all a hypothetical possibility, so it is drawn with dotted lines.


%%%%%%%%%%%%%%%%%%%%%%%%%%%%%%%%%%%%%%%%%
%%%%%%%%%%%%%%%%%%%%%%%%%%%%%%%%%%%%%%%%%
\section{Model the lolli elimination}

Now let us model the lolli elimination. To do that, let's first model the part where we introduce the ``$A$'':

\begin{diagram}

  \draw[densely dotted, fill=grey90] 
      (-0.15, 0.6) -- (0.85, 0.6) -- (0.85, -0.4) -- (-0.15, -0.4) -- (-0.15, 0.6);
  
  \node (j) [] at (0, 0) {
    \begin{prooftree}
      \hypo{\Proof/}
      \ellipsis{}{}
      \infer[rule style=no rule]1{A \lolli/ B}
      \hypo{}
      \infer1[\startrule/]{~~A^{a}~~}
      \infer2[\lolliElim/]{B}
      \infer1[\lolliIntro/$^{a}$]{A \lolli/ B}
    \end{prooftree}
  };

\end{diagram}

\noindent
To model that, we simply draw a new state, with an ``$A$'' in it. Let's set this new state over to the side, on the left of the hypothetical states:

\begin{diagram}

  % State 0
  \draw[densely dotted] (-1, -0.75) -- (1.25, -0.75) -- (1.25, 1.75) -- (-1, 1.75) -- (-1, -0.75);
  \coordinate[label=below:{\textbf{S}$_{0}$}] (s_0) at (0.175, -0.75);

    \node[] (a) [] at (0.15, 0.5) {$A$};

  % State 1
  \draw[densely dotted,spaced-arrows,->] (1.25, 0.5) -- (2.25, 0.5);
  \draw[densely dotted] (2.25, -0.75) -- (4.5, -0.75) -- (4.5, 1.75) -- (2.25, 1.75) -- (2.25, -0.75);
  \coordinate[label=below:{\textbf{S}$_{1}$}] (s_1) at (3.5, -0.75);

    \node[] (b) [] at (3.35, 0.5) {$B$};

  % State 3
  \draw[] (-4.25, -0.75) -- (-2, -0.75) -- (-2, 1.75) -- (-4.25, 1.75) -- (-4.25, -0.75);
  \coordinate[label=below:{\textbf{S}$_{2}$}] (s_1) at (-3.1, -0.75);

    \node[] (b) [] at (-3.1, 0.5) {$A$};

\end{diagram}

\noindent
Next let us model the elimination, i.e., the part of the proof tree that is highlighted here:

\begin{diagram}

  \draw[densely dotted, fill=grey90] 
      (-1.9, 0.4) -- (0.85, 0.4) -- (0.85, -0.4) -- (-0.25, -0.4) -- (-0.25, -0.75) -- (-0.8, -0.75) --
      (-0.8, -0.4) -- (-1.9, -0.4) -- (-1.9, 0.4);
  
  \node (j) [] at (0, 0) {
    \begin{prooftree}
      \hypo{\Proof/}
      \ellipsis{}{}
      \infer[rule style=no rule]1{A \lolli/ B}
      \hypo{}
      \infer1[\startrule/]{~~A^{a}~~}
      \infer2[\lolliElim/]{B}
      \infer1[\lolliIntro/$^{a}$]{A \lolli/ B}
    \end{prooftree}
  };

\end{diagram}

\noindent
Here we apply the ``$A$'' to the lolli ``$A \lolli/ B$.'' And that delivers a ``$B$.'' In our model, we can depict this by laying $S_{2}$ over the top of $S_{0}$, to show that the ``$A$'' in $S_{2}$ is being supplied for the hypothetical ``$A$'' in $S_{0}$:

\begin{diagram}

  % State 1
  \draw[densely dotted] (-1, -0.75) -- (1.25, -0.75) -- (1.25, 1.75) -- (-1, 1.75) -- (-1, -0.75);
  \coordinate[label=below:{\textbf{S}$_{1}$}] (s_1) at (0.175, -0.75);

    \node[] (a) [] at (0.15, 0.5) {$B$};

  % State 3
  \draw[densely dotted,spaced-arrows,->] (-2, 0.5) -- (-1, 0.5);  
  \draw[] (-4.25, -0.75) -- (-2, -0.75) -- (-2, 1.75) -- (-4.25, 1.75) -- (-4.25, -0.75);
  \coordinate[label=below:{\textbf{S}$_{2}$}] (s_2) at (-3.1, -0.75);

    \node[] (b) [] at (-3.1, 0.5) {$A$};

\end{diagram}

\noindent
And once ``$A$'' becomes actualized (rather than staying as a mere possibility), then the transition to $S_{1}$ is actualized too, so ``$B$'' becomes actualized in $S_{1}$ as well:


\begin{diagram}

  % State 1
  \draw[] (-1, -0.75) -- (1.25, -0.75) -- (1.25, 1.75) -- (-1, 1.75) -- (-1, -0.75);
  \coordinate[label=below:{\textbf{S}$_{1}$}] (s_1) at (0.175, -0.75);

    \node[] (a) [] at (0.15, 0.5) {$B$};

  % State 3
  \draw[spaced-arrows,->] (-2, 0.5) -- (-1, 0.5);  
  \draw[] (-4.25, -0.75) -- (-2, -0.75) -- (-2, 1.75) -- (-4.25, 1.75) -- (-4.25, -0.75);
  \coordinate[label=below:{\textbf{S}$_{2}$}] (s_2) at (-3.1, -0.75);

    \node[] (b) [] at (-3.1, 0.5) {$A$};

\end{diagram}

\noindent
At this point, we started with a lolli, and then we eliminated it.


%%%%%%%%%%%%%%%%%%%%%%%%%%%%%%%%%%%%%%%%%
%%%%%%%%%%%%%%%%%%%%%%%%%%%%%%%%%%%%%%%%%
\section{Model the re-introduction}

Now we are at the crucial point in the proof. The question is: we have eliminated the lolli, but do we have enough to re-introduce it?

In the proof tree, we saw that we can indeed do this. We can see it in the highlighted part here:

\begin{diagram}

  \draw[densely dotted, fill=grey90] 
      (-0.15, 0.6) -- (0.85, 0.6) -- (0.85, -0.4) -- (-0.25, -0.4) -- (-0.25, -0.65) -- 
      (0.2, -0.65) -- (0.2, -1.25) -- (-1.25, -1.25) -- (-1.25, -0.65) -- (-0.8, -0.65) --
      (-0.8, -0.15) -- (-0.15, -0.15) -- (-0.15, 0.6);
  
  \node (j) [] at (0, 0) {
    \begin{prooftree}
      \hypo{\Proof/}
      \ellipsis{}{}
      \infer[rule style=no rule]1{A \lolli/ B}
      \hypo{}
      \infer1[\startrule/]{~~A^{a}~~}
      \infer2[\lolliElim/]{B}
      \infer1[\lolliIntro/$^{a}$]{A \lolli/ B}
    \end{prooftree}
  };

\end{diagram}

\noindent
This says that, since we can see a path from ``$A$"' to ``$B$,'' we can cancel that part of the proof tree out, and encode it as a lolli.

In our model, we can see this too. In our model, we have exactly that: a path from ``$A$'' to ``$B$.'' So we can rewind the model, and then redraw the transition from $S_{2}$ to $S_{1}$ with dotted lines, to show that it is a hypothetical possibility:

\begin{diagram}

  % State 0
  \draw[densely dotted] (-1, -0.75) -- (1.25, -0.75) -- (1.25, 1.75) -- (-1, 1.75) -- (-1, -0.75);
  \coordinate[label=below:{\textbf{S}$_{2}$}] (s_2) at (0.175, -0.75);

    \node[] (a) [] at (0.15, 0.5) {$A$};

  % State 1
  \draw[densely dotted,spaced-arrows,->] (1.25, 0.5) -- (2.25, 0.5);
  \draw[densely dotted] (2.25, -0.75) -- (4.5, -0.75) -- (4.5, 1.75) -- (2.25, 1.75) -- (2.25, -0.75);
  \coordinate[label=below:{\textbf{S}$_{1}$}] (s_1) at (3.5, -0.75);

    \node[] (b) [] at (3.35, 0.5) {$B$};

\end{diagram}

\noindent
Now we are right back where we started. We have modeled the lolli ``$A \lolli/ B$,'' which says that if you give me an ``$A$'' I'll give you back a ``$B$.'' And that is precisely what our model shows: the hypothetical possibility that, if ``$A$'' were actualized, we would transition to a state where ``$B$'' would be actualized.


%%%%%%%%%%%%%%%%%%%%%%%%%%%%%%%%%%%%%%%%%
%%%%%%%%%%%%%%%%%%%%%%%%%%%%%%%%%%%%%%%%%
\section{The expansion}

Recall what we were trying to model:

$$
\begin{prooftree}
  \hypo{\Proof/}
  \ellipsis{}{}
  \infer[rule style=no rule]1{A \lolli/ B}
\end{prooftree}
\hskip 1cm\rightsquigarrow_{\eta}\hskip 1cm
\begin{prooftree}
  \hypo{\Proof/}
  \ellipsis{}{}
  \infer[rule style=no rule]1{A \lolli/ B}
  \hypo{}
  \infer1[\startrule/]{~~A^{a}~~}
  \infer2[\lolliElim/]{B}
  \infer1[\lolliIntro/$^{a}$]{A \lolli/ B}
\end{prooftree}
$$

\noindent
That describes a proof expansion. It says if we start with a lolli and then immediately eliminate it, we can then re-introduce the lolli.

If you look at the model we built, you can see that this is exactly what the model tells us. We modeled the situation where we start with a lolli, and then we eliminate it. We were left with a model that we can then introduce the lolli again.


%%%%%%%%%%%%%%%%%%%%%%%%%%%%%%%%%%%%%%%%%
%%%%%%%%%%%%%%%%%%%%%%%%%%%%%%%%%%%%%%%%%
\section{Summary}

In this chapter, we modeled the proof that the \lolliElim/ rule gives us enough information to reintroduce a lolli after we eliminate it. To show that, we modeled the situation where we first eliminated a lolli, then we re-introduced it. And we saw that the set of states we went through correspond exactly to the expanded proof tree we put together in the last chapter.


\end{document}
