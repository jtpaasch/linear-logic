\documentclass[../../../main.tex]{subfiles}
\begin{document}

%%%%%%%%%%%%%%%%%%%%%%%%%%%%%%%%%%%%%%%%%
%%%%%%%%%%%%%%%%%%%%%%%%%%%%%%%%%%%%%%%%%
%%%%%%%%%%%%%%%%%%%%%%%%%%%%%%%%%%%%%%%%%
\chapter{Nesting lollis}


%%%%%%%%%%%%%%%%%%%%%%%%%%%%%%%%%%%%%%%%%
%%%%%%%%%%%%%%%%%%%%%%%%%%%%%%%%%%%%%%%%%
\section{Layering lollis}

We can layer lolli introductions on top of each other. For example, consider this proof tree:

\begin{diagram}

  \draw[draw=black, densely dotted, fill=grey80]
      (-1.5, 0.3) -- (-1.5, -0.5) -- (0.5, -0.5) -- (0.5, -0.15) -- (3.1, -0.15) -- 
      (3.1, 1.15) -- (-0.15, 1.15) -- (-0.15, 0.3) -- (-1.5, 0.3);
  \draw[spaced-arrows,->] (3.1, 0.5) -- (3.85, 0.5) -- (3.85, -0.75) -- (1.5, -0.75);

  \node (j) [] at (0, 0) {
    \begin{prooftree}
      \hypo{} 
      \infer1[\startrule/]{b(s) :: true}
      \hypo{}
      \infer1[\startrule/]{t(s, p) :: true^{a}}
      \infer2[\rulename{trade}]{b(p) :: true}
      \infer1[\lolliIntro/$^{a}$]{t(s, p) \lolli/ b(p) :: true}
    \end{prooftree}
  };

\end{diagram}

\noindent
Here we derive the lolli ``$t(s, p) \lolli/ b(p) :: true$'' --- i.e., if you give me the option to trade squash for potatoes, I'll use that up and get potatoes in my basket. 

Now look at ``$b(s) :: true$'' and ``$t(s, p) \lolli/ b(p) :: true$.'' How are they related? Well, we can see that $b(s) :: true$ leads to ``$t(s, p) \lolli/ b(p) :: true$.'' So here we have another ``if---then'' relationship, that we can introduce as a lolli, if we so desire:

\begin{diagram}

  \draw[draw=black, densely dotted, fill=grey90]
    (-3.25, 1.5) -- (3.25, 1.5) -- (3.25, 0) -- (2, 0) -- (2, -0.8) --
    (-2.25, -0.8) -- (-2.25, 0.25) -- (-3.25, 0.25) -- (-3.25, 1.5);

  \draw[draw=black, densely dotted, fill=grey80]
      (-1.5, 0.6) -- (-1.5, -0.2) -- (0.5, -0.2) -- (0.5, 0.25) -- (3.1, 0.25) -- 
      (3.1, 1.35) -- (-0.15, 1.35) -- (-0.15, 0.6) -- (-1.5, 0.6);
  \draw[spaced-arrows,->] (3.1, 0.75) -- (3.85, 0.75) -- (3.85, -0.45) -- (1.5, -0.45);

  \node (j) [] at (0, 0) {
    \begin{prooftree}
      \hypo{} 
      \infer1[\startrule/]{b(s) :: true}
      \hypo{}
      \infer1[\startrule/]{t(s, p) :: true^{a}}
      \infer2[\rulename{trade}]{b(p) :: true}
      \infer1[\lolliIntro/$^{a}$]{t(s, p) \lolli/ b(p) :: true}
      \infer1[\lolliIntro/]{b(s) \lolli/ (t(s, p) \lolli/ b(p)) :: true}
    \end{prooftree}
  };

  \draw[spaced-arrows,->] (-3.25, 0.75) -- (-4, 0.75) -- (-4, -1) -- (-3, -1);
  
\end{diagram}

\noindent
Notice the layering or nesting here. There is already part of the tree that we've greyed out (the darker grey), in order to introduce the first lolli. Now we grey out even more of the tree, to introduce the second lolli.

Since we're deactivating the lighter greyed out part of the proof tree, we need to mark its top assumption with a unique superscript label, and we also need to put that same superscript on the bottom ``\lolliIntro/''. We already used a little ``$a$,'' so let's use a little ``$b$'':

\begin{diagram}

  \draw[draw=black, densely dotted, fill=grey90]
    (-3.25, 1.5) -- (3.25, 1.5) -- (3.25, 0) -- (2, 0) -- (2, -0.8) --
    (-2.25, -0.8) -- (-2.25, 0.25) -- (-3.25, 0.25) -- (-3.25, 1.5);
  \draw[spaced-arrows,->] (-3.25, 0.75) -- (-4, 0.75) -- (-4, -1) -- (-3, -1);

  \draw[draw=black, densely dotted, fill=grey80]
      (-1.5, 0.6) -- (-1.5, -0.2) -- (0.5, -0.2) -- (0.5, 0.25) -- (3.1, 0.25) -- 
      (3.1, 1.35) -- (-0.15, 1.35) -- (-0.15, 0.6) -- (-1.5, 0.6);
  \draw[spaced-arrows,->] (3.1, 0.75) -- (3.85, 0.75) -- (3.85, -0.45) -- (1.5, -0.45);

  \node (j) [] at (0, 0) {
    \begin{prooftree}
      \hypo{} 
      \infer1[\startrule/]{b(s) :: true^{b}}
      \hypo{}
      \infer1[\startrule/]{t(s, p) :: true^{a}}
      \infer2[\rulename{trade}]{b(p) :: true}
      \infer1[\lolliIntro/$^{a}$]{t(s, p) \lolli/ b(p) :: true}
      \infer1[\lolliIntro/$^{b}$]{b(s) \lolli/ (t(s, p) \lolli/ b(p)) :: true}
    \end{prooftree}
  };
  
\end{diagram}


%%%%%%%%%%%%%%%%%%%%%%%%%%%%%%%%%%%%%%%%%
%%%%%%%%%%%%%%%%%%%%%%%%%%%%%%%%%%%%%%%%%
\section{Just the superscript labels}

We can remove our dotted lines and arrow hints, so that we can see the tree as we normally write it:

\begin{prooftree*}
  \hypo{} 
  \infer1[\startrule/]{b(s) :: true^{b}}
  \hypo{}
  \infer1[\startrule/]{t(s, p) :: true^{a}}
  \infer2[\rulename{trade}]{b(p) :: true}
  \infer1[\lolliIntro/$^{a}$]{t(s, p) \lolli/ b(p) :: true}
  \infer1[\lolliIntro/$^{b}$]{b(s) \lolli/ (t(s, p) \lolli/ b(p)) :: true}
\end{prooftree*}

\noindent
Of course, it's much easier to see which part of the proof tree each lolli is derived from if we use dotted lines and arrows. But dotted lines and arrows are cumbersome to write or draw. It's much easier to simply use superscripts.

Besides, the superscripts do the job. The superscripted ``$a$'' and ``$b$'' in this proof tree tell us precisely which parts of the tree each lolli is derived from:

\begin{itemize}
  \item{At the point where the first lolli is introduced, we have the label ``\lolliIntro/$^{a}$.'' The superscripted ``$a$'' tells us that this lolli is derived from this introduction point, all the way up to the assumption that is also marked with a superscripted ``$a$.''}
  \item{At the point where the second lolli is introduced, we have the label ``\lolliIntro/$^{b}$.'' The superscripted ``$b$'' tells us that this second lolli is derived from its introduction point, all the way up to the assumption that is also marked with a subscripted ``$b$.''}
\end{itemize}


%%%%%%%%%%%%%%%%%%%%%%%%%%%%%%%%%%%%%%%%%
%%%%%%%%%%%%%%%%%%%%%%%%%%%%%%%%%%%%%%%%%
\section{The layered structure}

Notice that the first lolli (the one marked with a superscripted ``$a$'') is nested \emph{inside} the second lolli (the one marked with the superscripted ``$b$''). This makes sense. 

The second lolli was extracted from the proof tree after the first one was. So the second lolli will have embedded in it whatever we pull out of the tree to introduce it. And since the first lolli was in the tree when we pulled out the second lolli, we ended up with the first lolli embedded in the second one.

Indeed, just look at the structure of the second lolli judgment itself. Note where the antecedent and the consequent are, and notice how the first lolli is nested inside its consequent:

\begin{diagram}

  \node (j) [] at (0, 0) {$b(s) ~~~~ \lolli/ ~~~(~~~~t(s, p) \lolli/ b(p)~~~~) :: true$};

  \draw[densely dotted] (-0.75, -0.1) -- (-0.75, -0.35) -- (1.6, -0.35) -- (1.6, -0.1);

  \draw[spaced-arrows,densely dotted,->] (0.5, -1) -- (0.5, -0.35);
  \node (l_1) [label=below:{nested lolli}] at (0.5, -0.9) {};

  \draw (-1.25, 0.25) -- (-1.25, 0.5) -- (2.2, 0.5) -- (2.2, 0.25);
  \draw[spaced-arrows,->] (0.5, 1.25) -- (0.5, 0.5);
  \node (l_2) [label=above:{consequent}] at (0.5, 1.15) {};
    
  \draw (-3.5, 0.25) -- (-3.5, 0.5) -- (-2.25, 0.5) -- (-2.25, 0.25);
  \draw[spaced-arrows,->] (-2.9, 1.25) -- (-2.9, 0.5);
  \node (l_2) [label=above:{antecedent}] at (-2.9, 1.15) {};
    
\end{diagram}


%%%%%%%%%%%%%%%%%%%%%%%%%%%%%%%%%%%%%%%%%
%%%%%%%%%%%%%%%%%%%%%%%%%%%%%%%%%%%%%%%%%
\section{Rewinding the model}

Let us model the nested lollis, so we can see how this works at the level of the model.

First, let's draw the model at the point right before we introduce the first lolli. The proof tree at this point  is here:

\begin{prooftree*}
  \hypo{} 
  \infer1[\startrule/]{b(s) :: true^{b}}
  \hypo{}
  \infer1[\startrule/]{t(s, p) :: true^{a}}
  \infer2[\rulename{trade}]{b(p) :: true}
\end{prooftree*}

\noindent
Here's the model for that:

\begin{diagram}

  % State 0
  \draw (-1, -0.75) -- (1.25, -0.75) -- (1.25, 1.75) -- (-1, 1.75) -- (-1, -0.75);
  \coordinate[label=below:{\textbf{S}$_{0}$}] (s_0) at (0.175, -0.75);

    \node[o-point] (s) [label=below:{$s$}] at (-0.5, 1) {};
    \node[o-point] (p) [label=below:{$p$}] at (0.75, 0) {};

    \coordinate[label=above:{\fbox{$b$}}] (b) at (-0.5, 1);
    \draw[spaced-arrows,->] (s) to node [fill=white] {$t$} (p);

  % State 1
  \draw[spaced-arrows,->] (1.25, 0.5) -- (2.25, 0.5);
  \draw (2.25, -0.75) -- (4.5, -0.75) -- (4.5, 1.75) -- (2.25, 1.75) -- (2.25, -0.75);
  \coordinate[label=below:{\textbf{S}$_{1}$}] (s_1) at (3.5, -0.75);

    \node[o-point] (p_1) [label=below:{$p$}] at (4, 0) {};

    \node[draw] (b_1) at (4, 0.4) {$b$};

\end{diagram}

\noindent
Then, when we introduce the first lolli, we rewind the model back to the point where I only have squash in my basket:

\begin{diagram}

  % State 0
  \draw (-1, -0.75) -- (1.25, -0.75) -- (1.25, 1.75) -- (-1, 1.75) -- (-1, -0.75);
  \coordinate[label=below:{\textbf{S}$_{0}$}] (s_0) at (0.175, -0.75);

    \node[o-point] (s) [label=below:{$s$}] at (-0.5, 1) {};

    \coordinate[label=above:{\fbox{$b$}}] (b) at (-0.5, 1);

\end{diagram}

\noindent
Then we play out the first lolli, as a hypothetical situation (which we mark with dotted lines):

\begin{diagram}

  % State 0
  \draw (-1, -0.75) -- (1.25, -0.75) -- (1.25, 1.75) -- (-1, 1.75) -- (-1, -0.75);
  \coordinate[label=below:{\textbf{S}$_{0}$}] (s_0) at (0.175, -0.75);

    \node[o-point] (s) [label=below:{$s$}] at (-0.5, 1) {};
    \node[o-point] (p) [label=below:{$p$}] at (0.75, 0) {};

    \coordinate[label=above:{\fbox{$b$}}] (b) at (-0.5, 1);
    \draw[spaced-arrows,dotted,->] (s) to node [fill=white] {$t$} (p);

  % State 1
  \draw[spaced-arrows,dotted,->] (1.25, 0.5) -- (2.25, 0.5);
  \draw[densely dotted] (2.25, -0.75) -- (4.5, -0.75) -- (4.5, 1.75) -- (2.25, 1.75) -- (2.25, -0.75);
  \coordinate[label=below:{\textbf{S}$_{1}$}] (s_1) at (3.5, -0.75);

    \node[o-point] (p_1) [label=below:{$p$}] at (4, 0) {};

    \node[draw, densely dotted] (b_1) at (4, 0.4) {$b$};

\end{diagram}

\noindent
At this point, we have modeled the first lolli in our proof tree. 

Now we want to introduce the second lolli. We model this just as we modeled the first lolli. 

When we modeled the first lolli, the first thing we did was note which part of the proof tree we greyed out. Then we rewound our model to remove the parts that were greyed out. 

We need to do the same thing here too, for the second lolli. For the second lolli, we pulled out the more lightly greyed out part of the tree:

\begin{diagram}

  \draw[draw=black, densely dotted, fill=grey90]
    (-3.25, 1.5) -- (3.25, 1.5) -- (3.25, 0) -- (2, 0) -- (2, -0.8) --
    (-2.25, -0.8) -- (-2.25, 0.25) -- (-3.25, 0.25) -- (-3.25, 1.5);
  \draw[spaced-arrows,->] (-3.25, 0.75) -- (-4, 0.75) -- (-4, -1) -- (-3, -1);

  \draw[draw=black, densely dotted, fill=grey80]
      (-1.5, 0.6) -- (-1.5, -0.2) -- (0.5, -0.2) -- (0.5, 0.25) -- (3.1, 0.25) -- 
      (3.1, 1.35) -- (-0.15, 1.35) -- (-0.15, 0.6) -- (-1.5, 0.6);
  \draw[spaced-arrows,->] (3.1, 0.75) -- (3.85, 0.75) -- (3.85, -0.45) -- (1.5, -0.45);

  \node (j) [] at (0, 0) {
    \begin{prooftree}
      \hypo{} 
      \infer1[\startrule/]{b(s) :: true^{b}}
      \hypo{}
      \infer1[\startrule/]{t(s, p) :: true^{a}}
      \infer2[\rulename{trade}]{b(p) :: true}
      \infer1[\lolliIntro/$^{a}$]{t(s, p) \lolli/ b(p) :: true}
      \infer1[\lolliIntro/$^{b}$]{b(s) \lolli/ (t(s, p) \lolli/ b(p)) :: true}
    \end{prooftree}
  };
 
\end{diagram}

\noindent
So, we rewind our model back, to make sure that we remove these more lightly greyed out parts too. To do that, we remove the first lolli, which corresponds to the hypothetical situation drawn with dots. That leaves our model with just squash in my basket --- i.e., ``$b(s) :: true$'':

\begin{diagram}

  % State 0
  \draw (-1, -0.75) -- (1.25, -0.75) -- (1.25, 1.75) -- (-1, 1.75) -- (-1, -0.75);
  \coordinate[label=below:{\textbf{S}$_{0}$}] (s_0) at (0.175, -0.75);

    \node[o-point] (s) [label=below:{$s$}] at (-0.5, 1) {};

    \coordinate[label=above:{\fbox{$b$}}] (b) at (-0.5, 1);

\end{diagram}

\noindent
But since ``$b(s) :: true$'' gets greyed out at the stage where we introduce the second lolli, we need to remove that from our model as well. That leaves us with an empty initial state:

\begin{diagram}

  % State 0
  \draw (-1, -0.75) -- (1.25, -0.75) -- (1.25, 1.75) -- (-1, 1.75) -- (-1, -0.75);
  \coordinate[label=below:{\textbf{S}$_{0}$}] (s_0) at (0.175, -0.75);

\end{diagram}

\noindent
At this point, we have rewound our model all the way back to the start. We are ready to play out the second lolli, in order to double check that it is valid.


%%%%%%%%%%%%%%%%%%%%%%%%%%%%%%%%%%%%%%%%%
%%%%%%%%%%%%%%%%%%%%%%%%%%%%%%%%%%%%%%%%%
\section{Play out the lollis}

The second lolli is this:

\begin{equation*}
  b(s) \lolli/ (t(s, p) \lolli/ b(p)) :: true
\end{equation*}

\noindent
What is this lolli claiming? It is making a hypothetical claim: \emph{if} you have squash in your basket, \emph{then} this other lolli --- i.e., ``$t(s, p) \lolli/ b(p)$'' --- will be true.

To play this out, we play out the antecedent, hypothetically. The antecedent is ``$b(s)$'' --- i.e., I have squash in my basket. So let's suppose that, hypothetically, I have squash in my basket:

\begin{diagram}

  % State 0
  \draw (-1, -0.75) -- (1.25, -0.75) -- (1.25, 1.75) -- (-1, 1.75) -- (-1, -0.75);
  \coordinate[label=below:{\textbf{S}$_{0}$}] (s_0) at (0.175, -0.75);

    \node[o-point] (s) [label=below:{$s$}] at (-0.5, 1) {};

    \node[draw, densely dotted] (b_1) at (-0.5, 1.4) {$b$};

\end{diagram}

\noindent
Can we realize the consequent, from here? The consequent is ``$t(s, p) \lolli/ b(p)$'' --- i.e., it is itself a lolli. So we need to play out this one too.

First we play out the antecedent, which is ``$t(s, p)$.'' So, let's suppose that, hypothetically, I can trade squash for potatoes:

\begin{diagram}

  % State 0
  \draw (-1, -0.75) -- (1.25, -0.75) -- (1.25, 1.75) -- (-1, 1.75) -- (-1, -0.75);
  \coordinate[label=below:{\textbf{S}$_{0}$}] (s_0) at (0.175, -0.75);

    \node[o-point] (s) [label=below:{$s$}] at (-0.5, 1) {};
    \node[o-point] (p) [label=below:{$p$}] at (0.75, 0) {};

    \node[draw, densely dotted] (b_1) at (-0.5, 1.4) {$b$};
    \draw[spaced-arrows,dotted,->] (s) to node [fill=white] {$t$} (p);

\end{diagram}

\noindent
Now let's play out the consequent of ``$t(s, p) \lolli/ b(p)$,'' which is ``$b(p)$'' --- i.e., I will end up with potatoes in my basket. So, let's suppose that, hypothetically, I will end up with potatoes in my basket:

\begin{diagram}

  % State 0
  \draw (-1, -0.75) -- (1.25, -0.75) -- (1.25, 1.75) -- (-1, 1.75) -- (-1, -0.75);
  \coordinate[label=below:{\textbf{S}$_{0}$}] (s_0) at (0.175, -0.75);

    \node[o-point] (s) [label=below:{$s$}] at (-0.5, 1) {};
    \node[o-point] (p) [label=below:{$p$}] at (0.75, 0) {};

    \node[draw, densely dotted] (b_1) at (-0.5, 1.4) {$b$};
    \draw[spaced-arrows,dotted,->] (s) to node [fill=white] {$t$} (p);

  % State 1
  \draw[spaced-arrows,dotted,->] (1.25, 0.5) -- (2.25, 0.5);
  \draw[densely dotted] (2.25, -0.75) -- (4.5, -0.75) -- (4.5, 1.75) -- (2.25, 1.75) -- (2.25, -0.75);
  \coordinate[label=below:{\textbf{S}$_{1}$}] (s_1) at (3.5, -0.75);

    \node[o-point] (p_1) [label=below:{$p$}] at (4, 0) {};

    \node[draw, densely dotted] (b_1) at (4, 0.4) {$b$};

\end{diagram}

\noindent
At this point, we have played out the full lolli in our model. So now we can ask: is the lolli true in this situation? Is it true in this scenario that, if I were to get squash in my basket, then the following would hold: if I were then able to trade squash for potatoes, I could get potatoes? 

The answer is of course yes. We can see in the model that there is a direct path from having squash in my basket to having potatoes in my basket, if these hypotheticals get realized. So we can see that the second lolli ---

\begin{equation*}
  b(s) \lolli/ (t(s, p) \lolli/ b(p)) :: true
\end{equation*}

\noindent
--- is valid in this situation too.



%%%%%%%%%%%%%%%%%%%%%%%%%%%%%%%%%%%%%%%%%
%%%%%%%%%%%%%%%%%%%%%%%%%%%%%%%%%%%%%%%%%
\section{Summary}

Lollis can be nested. The rule for introducing a second lolli is no different than the rule for introducing a first lolli. It's just that the first lolli will get embedded into the second lolli. Similarly, we model a second lolli in just the same way that we model the first lolli. It's just that the situation is more complicated, and we have to be careful to model each step correctly.


\end{document}
