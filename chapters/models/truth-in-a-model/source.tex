\documentclass[../../../main.tex]{subfiles}
\begin{document}

%%%%%%%%%%%%%%%%%%%%%%%%%%%%%%%%%%%%%%%%%
%%%%%%%%%%%%%%%%%%%%%%%%%%%%%%%%%%%%%%%%%
%%%%%%%%%%%%%%%%%%%%%%%%%%%%%%%%%%%%%%%%%
\chapter{Truth in a model}


%%%%%%%%%%%%%%%%%%%%%%%%%%%%%%%%%%%%%%%%%
%%%%%%%%%%%%%%%%%%%%%%%%%%%%%%%%%%%%%%%%%
\section{Constructing sentences}

With a model, we can formulate sentences which express facts about the situation being modeled.

To construct such sentences, take any of the predicates in the model, and replace the $x$s, $y$s, and $z$s with objects from the model.

Then you can check if those sentences are true in the model, by checking if they describe a fact that truly holds in the model.


%%%%%%%%%%%%%%%%%%%%%%%%%%%%%%%%%%%%%%%%%
%%%%%%%%%%%%%%%%%%%%%%%%%%%%%%%%%%%%%%%%%
\section{Situation 1}

Recall from situation 1 that X23a is a part of J14, and X23b is too:

\begin{diagram}

  \node[o-point] (x23a) [label=left:{X23a}] at (-3, -1) {};
  \node[o-point] (j14) [label=above:{J14}] at (0, 1) {};
  \node[o-point] (x23b) [label=right:{X23b}] at (3, -1) {};

  \draw[spaced-arrows,->] 
    (x23a) to (-1.5, 0)
    to node [near start, fill=white] {part of} (-1.5, 0)
    to (j14);
    
  \draw[spaced-arrows,->]
    (x23b) to (1.5, 0)
    to node [near start, fill=white] {part of} (1.5, 0)
    to (j14);

\end{diagram}

\noindent
Written out compactly:

\begin{model}
  \modelobjects{X23a, X23b, J14}
  \modelrelations{
    \item{$x$ is a part of $y$: (X23a, J14), (X23b, J14)}
  }
\end{model}

\noindent
To construct statements about this model, we take the predicate ``$x$ is a part of $y$,'' and we replace $x$ and $y$ with the names of objects from the model. Some of these sentences will be true (in this model), and some will be false.

Here are some statements that are true in this model:

\begin{itemize}
  \item{X23a is a part of J14}
  \item{X23b is a part of J14}
\end{itemize}

\noindent
Here are some statements that are false in this model:

\begin{itemize}
  \item{J14 is a part of X23a}
  \item{J14 is a part of X23b}
  \item{X23a is a part of X23b}
  \item{X23b is a part of X23a}
\end{itemize}


%%%%%%%%%%%%%%%%%%%%%%%%%%%%%%%%%%%%%%%%%
%%%%%%%%%%%%%%%%%%%%%%%%%%%%%%%%%%%%%%%%%
\section{Situation 2}

Recall that $H20$ is composed of $H$ atoms and $O$ atoms, and $CO2$ is composed of $C$ atoms and $O$ atoms.

\begin{diagram}

  \node[o-point] (o) [label=below:{O}] at (0, -1) {};
  \node[o-point] (h) [label=below:{H}] at (-3, -1) {};
  \node[o-point] (c) [label=below:{C}] at (3, -1) {};

  \node[o-point] (h2o) [label=above:{H20}] at (-1.5, 1) {};
  \node[o-point] (co2) [label=above:{CO2}] at (1.5, 1) {};

  \draw
    (h2o) to node [fill=white] {composed of} (-1.5, 0);
  \draw[spaced-arrows,->] (-1.5, 0) to (h);
  \draw[spaced-arrows,->] (-1.5, 0) to (o);

  \draw
    (co2) to node [fill=white] {composed of} (1.5, 0);
  \draw[spaced-arrows,->] (1.5, 0) to (o);
  \draw[spaced-arrows,->] (1.5, 0) to (c);

\end{diagram}

\noindent
Written down compactly:

\begin{model}
  \modelobjects{H, O, C, H20, CO2}
  \modelrelations{
    \item{$x$ is composed of $y$ and $z$: (H20, H, O), (CO2, O, C)}
  }
\end{model}

\noindent
To construct statements about this model, we take the predicate ``$x$ is a composed of $y$,'' and we replace $x$, $y$, and $z$ with the names of objects from the model. Some of these sentences will be true (in this model), and some will be false.

Here are some statements that are true in this model:

\begin{itemize}
  \item{H20 is composed of H and O.}
  \item{CO2 is composed of O and C.}
\end{itemize}

\noindent
Here are some statements that are false in this model:

\begin{itemize}
  \item{H is composed of H20 and O.}
  \item{O is composed of C and H20.}
  \item{H20 is composed of C and H.}
  \item{CO2 is composed of H20 and C.}
\end{itemize}


%%%%%%%%%%%%%%%%%%%%%%%%%%%%%%%%%%%%%%%%%
%%%%%%%%%%%%%%%%%%%%%%%%%%%%%%%%%%%%%%%%%
\section{Situation 3}

Recall the vegetarians and vegan (with their one meat-eater friend).

\begin{diagram}

  \node[o-point] (s) [label=below:{Sheila}] at (0, 0) {};
  \node[o-point] (h) [label=below:{Hanover}] at (-3, -1) {};
  \node[o-point] (j) [label=below:{Jorge}] at (3, -1) {};
  \node[o-point] (d) [label=above:{Debby}] at (-1.5, 1) {};
  \node[o-point] (c) [label=above:{Camilla}] at (1.5, 1) {};

  \coordinate[label=above:{\fbox{vegetarian}}] (s_vegt) at (0, 0);
  \coordinate[label=above right:{\fbox{vegetarian}}] (h_vegt) at (-3, -1);

  \coordinate[label=above left:{\fbox{vegan}}] (j_vegn) at (3, -1);
  \coordinate[label=below right:{\fbox{vegan}}] (c_vegn) at (1.5, 1);

  \coordinate[label=below left:{\fbox{meat}}] (d_meat) at (-1.5, 1);
  
\end{diagram}

\noindent
Written down compactly: 

\begin{model}
  \modelobjects{Sheila, Hanover, Jorge, Camilla, Debby}
  \modelrelations{
    \item{$x$ is a vegetarian: Sheila, Hanover}
    \item{$x$ is a vegan: Jorge, Camilla}
    \item{$x$ is a meat eater: Debby}
  }
\end{model}

\noindent
To construct statements about this model, we take the predicates ``$x$ is a vegetarian,'' ``$x$ is a vegan,'' and $x$ is a meat eater,'' and we replace $x$ with the names of objects from the model. Some of these sentences will be true (in this model), and some will be false.

Here are some statements that are true in this model:

\begin{itemize}
  \item{Sheila is a vegetarion.}
  \item{Jorge is a vegan.}
  \item{Debby is a meat eater.}
\end{itemize}

\noindent
Here are some statements that are false in this model:

\begin{itemize}
  \item{Hanover is a vegan.}
  \item{Camilla is a meat eater.}
  \item{Sheila is a vegan.}
  \item{Debby is a vegetarion.}
\end{itemize}


%%%%%%%%%%%%%%%%%%%%%%%%%%%%%%%%%%%%%%%%%
%%%%%%%%%%%%%%%%%%%%%%%%%%%%%%%%%%%%%%%%%
\section{Situation 4}

Recall how Keven and Jay work for Thecla, while Bea and Jo are self-employed:

\begin{diagram}

  \node[o-point] (j) [label=below:{Jay}] at (0, -1) {};
  \node[o-point] (k) [label=below:{Kevin}] at (-3, -1) {};
  \node[o-point] (b) [label=below:{Bea}] at (3, -1) {};
  \node[o-point] (t) [label=above:{Thecla}] at (-1.5, 1) {};
  \node[o-point] (l) [label=above:{Jo}] at (1.5, 1) {};

  \draw[spaced-arrows,->] (t) to (k);
  \draw[spaced-arrows,->] (t) to (j);
  
  \path[spaced-arrows,->] (l) edge[loop below, out=0, looseness=40] (l);
  \path[spaced-arrows,->] (b) edge[loop above, out=180, looseness=40] (b);

\end{diagram}

\noindent
Written out compactly:

\begin{model}
  \modelobjects{Kevin, Jay, Thecla, Bea, Jo}
  \modelrelations{
    \item{$x$ employs $y$: (Thecla, Kevin), (Thecla, Jay), (Bea, Bea), (Jo, Jo)}
  }
\end{model}

\noindent
To construct statements about this model, we take the predicate ``$x$ employs $y$,'' and we replace $x$ and $y$ with the names of objects from the model. Some of these sentences will be true (in this model), and some will be false.

Here are some statements that are true in this model:

\begin{itemize}
  \item{Thecla employs Kevin.}
  \item{Thecla employs Jay.}
  \item{Bea employs Bea.}
\end{itemize}

\noindent
Here are some statements that are false in this model:

\begin{itemize}
  \item{Kevin employs Kevin.}
  \item{Thecla employs Bea.}
  \item{Bea employs Jo.}
\end{itemize}


%%%%%%%%%%%%%%%%%%%%%%%%%%%%%%%%%%%%%%%%%
%%%%%%%%%%%%%%%%%%%%%%%%%%%%%%%%%%%%%%%%%
\section{Summary}

With a model, we can construct sentences that express facts about the situation being modeled. To construct such sentences, take any of the predicates in the model and replace the $x$s, $y$s, and $z$s with names of objects in the model. Then you can check if those statements are true or false in the situation being modeled.


\end{document}
