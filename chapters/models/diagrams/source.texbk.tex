\documentclass[../../../main.tex]{subfiles}
\begin{document}

%%%%%%%%%%%%%%%%%%%%%%%%%%%%%%%%%%%%%%%%%
%%%%%%%%%%%%%%%%%%%%%%%%%%%%%%%%%%%%%%%%%
%%%%%%%%%%%%%%%%%%%%%%%%%%%%%%%%%%%%%%%%%
\chapter{Modeling situations}


%%%%%%%%%%%%%%%%%%%%%%%%%%%%%%%%%%%%%%%%%
%%%%%%%%%%%%%%%%%%%%%%%%%%%%%%%%%%%%%%%%%
\section{Models}

A model is like an imagined version of a situation. To construct a model, we imagine a situation, and then we write it down. Of course, we don't write down every detail. We only write down the pieces we care about.

A model is convenient because we can reason about it. We can examine the model and figure out what's true in that model.


%%%%%%%%%%%%%%%%%%%%%%%%%%%%%%%%%%%%%%%%%
%%%%%%%%%%%%%%%%%%%%%%%%%%%%%%%%%%%%%%%%%
\section{A situation}

Consider the following situation:

\begin{quote}
  Bob and Geraldine are married, and they love each other deeply. The problem is, Geraldine's boss --- who's name is Julianne --- is also smitten with Bob.
\end{quote}

\noindent
Before going on, pause for a moment to think about this situation. If you had to write this down in a compact way, how would you do it? Who/what is involved (can you make a list), and how are these things related? If you had to draw a diagram of the situation, what would your diagram look like?



%%%%%%%%%%%%%%%%%%%%%%%%%%%%%%%%%%%%%%%%%
%%%%%%%%%%%%%%%%%%%%%%%%%%%%%%%%%%%%%%%%%
\section{The objects in a model}

To construct a model of some situation, we first write down all the entities that are involved in the situation. We call these entities the \vocab{objects} in the model. 

In the above scenario, there are three objects in play:

\begin{itemize}
  \item{Bob}
  \item{Geraldine}
  \item{Julianne}
\end{itemize}

\noindent
On a piece of paper, draw a dot for each of these, and label it. The dots can be anywhere on the paper. They just need to be spread out a little so we can draw lines between them. Something like this:

\begin{diagram}

  \node[o-point] (b) [label=below:{Bob}] at (-2, -2) {};
  \node[o-point] (g) [label=above right:{Geraldine}] at (0, 0) {};
  \node[o-point] (j) [label=above:{Julianne}] at (-4, 2) {};

\end{diagram}


%%%%%%%%%%%%%%%%%%%%%%%%%%%%%%%%%%%%%%%%%
%%%%%%%%%%%%%%%%%%%%%%%%%%%%%%%%%%%%%%%%%
\section{Relations in a model}

Next, we want to identify the different ways these objects are \vocab{related}. In the above scenario, some of the players love others. For example, Geraldine loves Bob. To diagram this, draw an arrow from Geraldine to Bob, and label it ``loves'':

\begin{diagram}

  \node[o-point] (b) [label=below:{Bob}] at (-2, -2) {};
  \node[o-point] (g) [label=above right:{Geraldine}] at (0, 0) {};
  \node[o-point] (j) [label=above:{Julianne}] at (-4, 2) {};

  \draw[spaced-arrows,->] 
    (g) to [out=0, in=30] (1, -1) 
    to node [near start, fill=white] {loves} (1, -1) 
    to [out=230, in=0] (b);
    
\end{diagram}

\noindent
Bob loves Geraldine back, so draw another arrow for that:

\begin{diagram}

  \node[o-point] (b) [label=below:{Bob}] at (-2, -2) {};
  \node[o-point] (g) [label=above right:{Geraldine}] at (0, 0) {};
  \node[o-point] (j) [label=above:{Julianne}] at (-4, 2) {};

  \draw[spaced-arrows,->]
    (b) to [out=120, in=90] (-2, -1)
    to node [near start, fill=white] {loves} (-2, -1)
    to [out=60, in=180] (g);
  \draw[spaced-arrows,->] 
    (g) to [out=0, in=30] (1, -1) 
    to node [near start, fill=white] {loves} (1, -1) 
    to [out=230, in=0] (b);

\end{diagram}

\noindent
Also, Julianne loves Bob, so add a third arrow for that:

\begin{diagram}

  \node[o-point] (b) [label=below:{Bob}] at (-2, -2) {};
  \node[o-point] (g) [label=above right:{Geraldine}] at (0, 0) {};
  \node[o-point] (j) [label=above:{Julianne}] at (-4, 2) {};

  \draw[spaced-arrows,->]
    (b) to [out=120, in=90] (-2, -1)
    to node [near start, fill=white] {loves} (-2, -1)
    to [out=60, in=180] (g);
  \draw[spaced-arrows,->] 
    (g) to [out=0, in=30] (1, -1) 
    to node [near start, fill=white] {loves} (1, -1) 
    to [out=230, in=0] (b);
  \draw[spaced-arrows,->] 
    (j) to [out=180, in=130] (-5, -.75) 
    to [out=-45, in=180] node [near start, fill=white] {loves} 
    (b);

\end{diagram}

\noindent
Does Bob love Julianne? No, he does not, so there is an arrow from Julianne to Bob, but there is no reciprocal arrow.


%%%%%%%%%%%%%%%%%%%%%%%%%%%%%%%%%%%%%%%%%
%%%%%%%%%%%%%%%%%%%%%%%%%%%%%%%%%%%%%%%%%
\section{More relations}

Another connection in the above scenario is this: Julianne is Geraldine's boss. To diagram this, draw an arrow from Julianne to Geraldine and label it ``employs'':

\begin{diagram}

  \node[o-point] (b) [label=below:{Bob}] at (-2, -2) {};
  \node[o-point] (g) [label=above right:{Geraldine}] at (0, 0) {};
  \node[o-point] (j) [label=above:{Julianne}] at (-4, 2) {};

  \draw[spaced-arrows,->]
    (b) to [out=120, in=90] (-2, -1)
    to node [near start, fill=white] {loves} (-2, -1)
    to [out=60, in=180] (g);
  \draw[spaced-arrows,->] 
    (g) to [out=0, in=30] (1, -1) 
    to node [near start, fill=white] {loves} (1, -1) 
    to [out=230, in=0] (b);
  \draw[spaced-arrows,->] 
    (j) to [out=180, in=130] (-5, -.75) 
    to [out=-45, in=180] node [near start, fill=white] {loves} 
    (b);

  \draw[spaced-arrows,->]
    (j) to [out=0] (-2, 1.5)
    to node [near start, fill=white] {employs} (-2, 1.5)
    to (g);

\end{diagram}


%%%%%%%%%%%%%%%%%%%%%%%%%%%%%%%%%%%%%%%%%
%%%%%%%%%%%%%%%%%%%%%%%%%%%%%%%%%%%%%%%%%
\section{Writing it down}

The above diagram is a model of the situation we described at the beginning of the chapter. Diagrams are nice because they are easy to understand.

However, we can also write the model down in a compact way. To do that, we first list the objects in the model:

\begin{model}
  \modelobjects{Geraldine, Julianne, Bob}
\end{model}

\noindent
Then we list the relations. So start a list:

\begin{model}
  \modelobjects{Geraldine, Julianne, Bob}
  \item{\textsc{Relations}: \ldots}
\end{model}

\noindent
Now we want to write down each relation that we modeled in the diagram.

The first relation we modeled is love. Rather than calling this by the rather generic name ``love,'' let's be more exact. Let's refer to this with the expression ``$x$ loves $y$'':

\begin{model}
  \modelobjects{Geraldine, Julianne, Bob}
  \modelrelations{
    \item{$x$ loves $y$: \ldots}
  }
\end{model}

\noindent
The expression ``$x$ loves $y$'' is called a \vocab{predicate}. A predicate is simply a description that has holes or slots in it that can be filled in with names. In this case, $x$ and $y$ are placeholders that can be filled in with names from the model. 

The next step is to list out each lover-lovee pair that satisfies the description/predicate ``$x$ loves $y$''. In our diagram above, Geraldine loves Bob, Bob loves Geraldine, and Julianne loves Bob. So:

\begin{model}
  \modelobjects{Geraldine, Julianne, Bob}
  \modelrelations{
    \item{$x$ loves $y$: (Geraldine, Bob), (Bob, Geraldine), (Julianne, Bob)}
  }
\end{model}

\noindent
Now we can turn to the second relation we modeled, namely employment. Rather than calling this by the rather generic name ``employment,'' let's again use a predicate: let's refer to this by the expression ``$x$ employs $y$'':


\begin{model}
  \modelobjects{Geraldine, Julianne, Bob}
  \modelrelations{
    \item{$x$ loves $y$: (Geraldine, Bob), (Bob, Geraldine), (Julianne, Bob)}
    \item{$x$ employs $y$: \ldots}
  }
\end{model}

\noindent
And then, list out all the employer-employee pairs that satisfy the description/predicate ``$x$ employs $y$'':

\begin{model}
  \modelobjects{Geraldine, Julianne, Bob}
  \modelrelations{
    \item{$x$ loves $y$: (Geraldine, Bob), (Bob, Geraldine), (Julianne, Bob)}
    \item{$x$ employs $y$: (Julianne, Geraldine)}
  }
\end{model}

\noindent
And with that, we are done. We have now written down all the same information we can see in the diagram. 

In fact, if we never saw the diagram in the first place, and were instead given this written description of the model, we could reconstruct the diagram ourselves. We could take a piece of paper and make a dot for each of the objects in the model. Then we could draw an arrow (labeled with ``loves'') for each of lover-lovee pairs, and we could draw an arrow (labeled with ``employs'') for each of the employer-employee pairs.

As an exercise you should try this. Take the written description we just put together, and draw a diagram from it.


%%%%%%%%%%%%%%%%%%%%%%%%%%%%%%%%%%%%%%%%%
%%%%%%%%%%%%%%%%%%%%%%%%%%%%%%%%%%%%%%%%%
\section{Summary}

To model a situation, you can diagram it. To do that, draw a dot for each object in the situation and label it with the name of the object. Then, draw arrows (and label them) to represent each relationship.

In addition, you can write down the model in a very compact way by listing all the objects, and all the objects that are connected through a relationship.

\end{document}
